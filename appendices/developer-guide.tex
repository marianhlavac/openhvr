\chapter{Developer Guide}\label{devguide}

This chapter briefly describe how developers can extend the functionalities
of the OpenHVR system, or use the Unity plug-in to fit the effects 
into their virtual reality applications and games.

To understand some of the server code, an experience with the Beego framework
is required. To learn more about the Beego framework, visit the
\href{https://beego.me/docs/intro/}{Beego Documentation} on their official
website.\footnote{\href{https://beego.me}{https://beego.me} is the official Beego framework website.}

\section*{Writing new device driver}

Directory \verb|openhvr-server/devicedrivers| contains implementations of
device drivers. To add new drives, create a new file with a new function,
that conform to following interface:

\begin{verbatim}
    package devicedrivers

    import "github.com/mmajko/openhvr-server/models"

    func FooDriver(device *models.Device, request *models.EffectRequest) error {
        ...
    }
\end{verbatim}

Implemented driver then must be registered in file \verb|openhvr-server/main.go|
on line 26. Example:

\begin{verbatim}
    ...
    devicedrivers.RegisterDriver("name_of_the_driver", devicedrivers.FooDriver)
    ...
\end{verbatim}

\section*{Server API documentation}

The Beego framework can automatically generate API developer documentation.
To read the documentation, run the developer server using the \verb|bee| command
(can be installed using command \verb|go get -u github.com/beego/bee|)

Run the developement server with the \verb|-downdoc=true -gendoc=true| arguments

\begin{verbatim}
    bee run -downdoc=true -gendoc=true
\end{verbatim}

then visit the address \verb|\href{http://localhost:47023/docs}{http://localhost:47023/docs}| to read the
documentation.

The generated documentation can be also found on the enclosed CD of the thesis.

\section*{Using the Unity plug-in}

Download the Unity package file \verb|openhvr.unitypackage| from the GitHub
repository or enclosed CD.

In Unity Editor, open contextual menu and select 
\verb|Assets| > \verb|Import Package| > \verb|Custom package...| and select
the \verb|openhvr.unitypackage| file.

In the scene, where the OpenHVR effects are desired to be used, create an empty
game object and add a new \verb|OpenHVR Manager| component to it.

Test the functionality by turning on the debug mode in the manager and running
the game. In the console a message will appear, reporting if the connection
to the OpenHVR server is successful.

To display OpenHVR devices and access their locations from in-game, attach
the \verb|OpenHVR Devices Enumerator| component to any game object.

To run a OpenHVR effect, attach the \verb|OpenHVR Effect Source| component to
any game object. For each effect you can set the following properties:

\begin{description}
    \item[Duration] The duration of the effect in seconds.
    \item[Effect Type] Type of the effect to produce.
    \item[Range] Determines the size of the sphere representing the effect range.
    If you want to turn on multiple devices at once, you can make the range larger.
    \item[Directional] If turned on, the effect will be directional and will
    be applied only to devices in the correct direction.
    \item[Play on Awake] The effect will start when the game object is created.
\end{description}

The position and direction of the effect source is determined by the values
present in the \verb|Transform| component of your game object.

To control the effect programatically, a following public methods in the
\verb|OpenHVREffect| class can be accessed:

\begin{description}
    \item[void Play()] Starts the effect for the specified duration.
    \item[void Cancel()] Immediately cancel the effect.
    \item[bool IsPlaying()] Returns \verb|true| if the effect is currently playing.
\end{description}