\chapter{Install Guide}

The purpose of this chapter is to guide administrators through the process
of instalation of the OpenHVR system.

\section*{Server installation}

Users are offered two ways to install OpenHVR server. Installation of the
server contains the configuration tool and is not needed to install separately.

\subsection*{Install using Docker image}

The easiest way to install OpenHVR server is to run Docker image, which is
publicly available in the Docker Hub\footnote{https://hub.docker.com/}
container registry. The requirement is to have Docker 
Engine\footnote{https://docs.docker.com/get-docker/} of version 2.2 or
newer installed on your computer. 

Run the image using the following command:

\begin{verbatim}
    docker run -d --mount \
    source=od,destination=/go/src/github.com/mmajko/openhvr-server/_data \
    -p 47023:47023 marianhlavac/openhvr
\end{verbatim}

The command will run the server on port 47023, and the configuration tool will
be available at \verb|http://localhost:47023|

\subsection*{Install manually}

If you do not prefer Docker, you can compile the server manually and run it.

You will need to install the following requirements:

\begin{itemize}
    \itemsep0em
    \item Go Tool command-line utility (version 1.13 or newer)
    \item Node.js JavaScript runtime (version 13.11 or newer) 
    \item NPM package manager (version 6.14 or newer)
\end{itemize}

To learn how to obtain an installation of Go tools, refer to the
\hyperlink{https://golang.org/doc/install}{Installation Instructions} at the
official Go website.\footnote{https://golang.org is the official Go website.}

Navigate to the directory containing the OpenHVR server source code named
\verb|openhvr-server|. In the directory run the following command to build the
server executable:

\begin{verbatim}
    go build main.go
\end{verbatim}

The library dependencies required for the build will be automatically downloaded.
After the build process is finished, you can find an executable in the current
directory.

Go to the \verb|configurator-tool| directory, which can be found in the current
working directory, and install the configurator tool dependencies using the npm's
\verb|install| command,

\begin{verbatim}
    npm install
\end{verbatim}

and build it with following command:

\begin{verbatim}
    npm run build
\end{verbatim}

The built files will appear in the \verb|./public/build| directory.

Before running the server, make sure the \verb|_data| directory exists inside
the \verb|openhvr-server| directory, and run the compiled executable.

\begin{verbatim}
    ./main
\end{verbatim}

\section*{Effect device preparation}

Before your electrical appliance can be used for producing the effects
within the OpenHVR system, it needs to be configured first.

\textbf{NOTE:} Make sure that the device is connected properly. Only a qualified
person should be allowed to install electronics working with higher voltages, to avoid
damaging the equipment or other risks.

\subsection*{Flash the Tasmota firmware}

The device must be flashed with the 
Tasmota\footnote{https://github.com/arendst/Tasmota/releases}
firmware on the chip of the device.
The appliance itself must be based on the ESP8266 chip, or connected to a 
smart relay or switch, which is based on the ESP8266 chip. To check if your
device is compatible with Tasmota firmware, consult the 
\hyperlink{https://templates.blakadder.com}{Tasmota Devices Templates Repository}
which lists all supported devices.

\textbf{TIP:} Some devices are difficult to flash, requires soldering on the
pins of the chip or other sophisticated methods of flashing. If you are not
experienced with electronics, leave the flashing process to someone
more experienced or try searching for a pre-flashed devices on the market.

\subsection*{Add the device to OpenHVR}

After your device is ready and running the Tasmota firmware, it needs to be
connected to the same LAN as the OpenHVR server. Determine its IP address and
open OpenHVR configuration tool.

Press the \textit{Add Device} button, select the type of effects this device
will be producing, and enter the connection details of your device into
the form. If you do not yet know the position and rotation of the device, you
can keep the fields empty. Select appropriate driver for controlling the
appliance. Specify connection parameter, if neccessary. If you have multi-channel
Tasmota device, the parameter must be equal to the channel name 
(e.g., \verb|Power1| for the first channel, \verb|Power2| for the second one).