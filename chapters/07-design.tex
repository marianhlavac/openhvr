\chapter{Designing the System}

The name "OpenHVR" was chosen as a codename and project name, under which
all source codes will be publicly available. The name emphasizes the open-source
essence of the project, and HVR stands for "Haptics in VR". A simple logo, to
go with the name, was also designed.


\begin{figure}[h]{}
\centering\includegraphics[width=2.5truein]{assets/openhvrlogo.png}
\caption{}

\end{figure}

\hypertarget{x-components-cooperation}{\section{Components cooperation}}
The following simple diagram is picturing the system components and the
communication between them.


\begin{figure}[h]{}
\centering\includegraphics[width=2.5truein]{assets/openhvr-diagram.pdf}
\caption{}

\end{figure}


\hypertarget{x-configurator-tool-wireframe-prototype}{\section{Configurator Tool wireframe prototype}}
As a first step in designing the Configurator Tool, a wireframe prototype
has been designed. Configurator Tool user interface will consist of only
two interactive screens — the main screen and screen for adding a new device.
On the main screen, the user will be able to see an overview of the room configuration,
manage (move, update, remove) effect devices, and will see the system’s status.
The "New device screen" will be displayed after clicking on the "Add device"
button at the main screen.


The prototype pictures layout of the user interface.


\begin{figure}[h]{}
\centering\includegraphics[width=2.5truein]{assets/wireframe-1-dash.pdf}
\caption{}

\end{figure}

\begin{figure}[h]{}
\centering\includegraphics[width=2.5truein]{assets/wireframe-2-editing.pdf}
\caption{}

\end{figure}

\begin{figure}[h]{}
\centering\includegraphics[width=2.5truein]{assets/wireframe-3-adding.pdf}
\caption{}

\end{figure}

\hypertarget{x-rest-api-design}{\section{REST API Design}}
Following is a list of resources available at the server’s API endpoint.
With short descriptions it represents a simple draft of functions provided 
by the server component.
Concrete resource and schemas documentation derived
directly from the implementation, is present
in \hyperlink{./13-developer-guide#server-api}{} of the
Developer Guide chapter.


\begin{description}

\item[GET /devices/]Returns a list of registered effect devices in the current room configuration,
together with their properties (e.g., location, rotation, name).

\item[POST /devices/]Registers a new effect device in the room.

\item[GET /devices/drivers]Returns a list of currently available drivers. New drivers can be added only
by extending the source code of the server.

\item[GET /devices/{deviceId}]Returns details about a specific device.

\item[PUT /devices/{deviceId}]Updates specific device properties (e.g., location, rotation, name).

\item[DELETE /devices/{deviceId}]Removes registration of the device from the room configuration.

\item[POST /effects/]Requests a new effect performance. This resource is meant to be requested by the
client each time there is an event happening in the VR scene, which should also
produce an accompanying effect in the room. The parameters sent by the client
(i.e., type of the effect, duration, location and direction, range) are used
by the server to calculate which effect devices need to be instructed and sends
the instructions to them.

\item[DELETE /effects/]Immediately cancels all previously requested effect performances. Useful when
the scene needs to be cleared, or, for any reason, it needs to be reassured that
no devices are producing any effects (for example, when quitting the VR
application).

\item[DELETE /effects/{effectId}]Immediately cancels a specific, previously requested effect performance.
Canceling specific performances is needed only when infinite duration
was specified when requesting. Otherwise, the effect will end after the
duration, and there is no need to cancel it.

\item[GET /effects/types]Returns list of available types of effects. Client applications can request
this list at their startup and raise a warning at the client-side if VR
application is requesting an effect that is not available.

\item[GET /system/status]Returns system status.

\end{description}


\hypertarget{x-api-security}{\section{API Security}}
The server will not support authentication, and all resources will be publicly
available. It is expected that each server will serve a single instance of
a virtual reality set-up and will run on a local network without public access.
Should the user ever be concerned about unwanted access to the system, the server
will support IP whitelisting
\footnote{IP Whitelisting is a common name for the process of filtering requests by the sender IP address and letting through only the requests with senders listed on the "whitelist".}.


The security of each of the device depends on the software and firmware provided
with the smart device itself. Drivers included in the OpenHVR server should
support secure connection too. Tasmota firmware supports username and
password authentication.
