\chapter{Designing the System}\label{design}

The name `OpenHVR' was chosen as a codename and project name, under which
all source codes will be publicly available. A simple logo to
go with the name, was also designed.

\begin{figure}[h]{}
\centering\includegraphics[width=0.5\textwidth]{assets/openhvrlogo.png}
\caption{OpenHVR logo}
\end{figure}

\hypertarget{x-components-cooperation}{\section{Components cooperation}}
The following simple diagram is picturing the system components and the
communication between them. There are two actors and five system
components defined. Administrator defines the configuration with the
Configurator Tool using his browser. All three client components
(i.e., Configurator Tool, Unity Plug-in, Effect Devices) are then communicating
with the server using REST HTTP API. In the future, the Effect Devices
can communicate using MQTT or UDP.


\begin{figure}[h]{}
\centering\includegraphics[width=\textwidth]{assets/openhvr-diagram.pdf}
\caption{Simple diagram picturing the overview of system functionality}
\begin{enumerate}
    \item Administrator is a user that configures the system and performs the first setup.
    \item VR User is a user experiencing VR application. This user can the same as the Administrator.
    \item Configurator Tool (as described in Section~\ref{analysis:conftool}).
    \item VR Application written in Unity Engine (or possibly later other supported frameworks and game engines) which use OpenHVR Unity Plug-in
    \item Unity plug-in (as described in Section~\ref{analysis:plugin}).
    \item Effect Devices registered in setup instance of OpenHVR System.
    \item OpenHVR Communication Server (as described in Section~\ref{analysis:server}).
\end{enumerate}
\end{figure}


\hypertarget{x-configurator-tool-wireframe-prototype}{\section{Configurator Tool wireframe prototype}}
As a first step in designing the Configurator Tool, a wireframe prototype
has been designed. Configurator Tool user interface will consist of only
two interactive screens — the main screen and screen for adding a new device.
On main screen (fig.\,\ref{wire1}), the user will be able to see an overview of the room configuration,
manage (update or remove) effect devices (as seen in fig.\,\ref{wire2}), and will see the system’s status.
The new device screen (fig.\,\ref{wire3}) will be displayed after clicking on the `Add device'
button at the main screen.


\begin{figure}[p]{}
\centering\includegraphics[width=\textwidth]{assets/wireframe-1-dash.pdf}
\caption{Wireframe design of the Configurator Tool's dashboard (main) screen}
\label{wire1}
\end{figure}

\begin{figure}[p]{}
\centering\includegraphics[width=\textwidth]{assets/wireframe-2-editing.pdf}
\caption{Wireframe design of the in-place device edit mode}
\label{wire2}
\end{figure}

\begin{figure}[p]{}
\centering\includegraphics[width=\textwidth]{assets/wireframe-3-adding.pdf}
\caption{Wireframe design of the `Add device' screen}
\label{wire3}
\end{figure}

\pagebreak


\hypertarget{x-rest-api-design}{\section{REST API Design}}
Following is a list of resources available at the server’s API endpoint.
With short descriptions it represents a simple draft of functions provided 
by the server component and doesn't represent the final form of the API.
Concrete resource and schemas documentation can be generated directly
from the implementation. The instructions how to generate the documentation
are present in \nameref{devguide} on page \pageref{devguide}.

\begin{description}

\item[GET /devices/]Returns a list of registered effect devices in the current room configuration,
together with their properties (e.g., location, rotation, name).

\item[POST /devices/]Registers a new effect device in the room.

\item[GET /devices/drivers]Returns a list of currently available drivers. New drivers can be added only
by extending the source code of the server.

\item[GET /devices/\{deviceId\}]Returns details about a specific device.

\item[PUT /devices/\{deviceId\}]Updates specific device properties (e.g., location, rotation, name).

\item[DELETE /devices/\{deviceId\}]Removes registration of the device from the room configuration.

\item[POST /effects/]Requests a new effect performance. This resource is meant to be requested by the
client each time there is an event happening in the VR scene, which should also
produce an accompanying effect in the room. The parameters sent by the client
(i.e., type of the effect, duration, location and direction, range) are used
by the server to calculate which effect devices need to be instructed and sends
the instructions to them.

\item[DELETE /effects/]Immediately cancels all previously requested effect performances. Useful when
the scene needs to be cleared, or, for any reason, it needs to be reassured that
no devices are producing any effects (for example, when quitting the VR
application).

\item[DELETE /effects/\{effectId\}]Immediately cancels a specific, previously requested effect performance.
Canceling specific performances is needed only when infinite duration
was specified when requesting. Otherwise, the effect will end after the
duration, and there is no need to cancel it.

\item[GET /effects/types]Returns list of available types of effects. Client applications can request
this list at their startup and raise a warning if the VR
application is requesting an effect that is not available.

\item[GET /system/status]Returns system status.

\end{description}


\hypertarget{x-api-security}{\section{API Security}}
The server will not support authentication, and all resources will be publicly
available. It is expected that each server will serve a single instance of
a virtual reality setup and will run on a local network without public access.
Should the user ever be concerned about unwanted access to the system, the server
will support IP whitelisting\footnote{IP Whitelisting is a common name for the process of filtering requests by the sender IP address and letting through only the requests with senders listed on the ``whitelist''.}.


The security of each of the device depends on the software and firmware provided
with the smart device itself. Drivers included in the OpenHVR server should
support secure connection too. Tasmota firmware supports username and
password authentication.
