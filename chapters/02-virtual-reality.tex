\chapter{Virtual reality}

Many definitions describe the term "virtual reality" (shortly VR) differently.
Some of them are older and define virtual reality as follows:

\begin{quotation}
Virtual reality is electronic simulation of environments experienced via
head mounted eye goggles and wired clothing enabling the end user to interact
in realistic three-dimensional situations (Coates, 1992)
\end{quotation}

Newer definitions define VR as:


\begin{quotation}
The use of computer technology to create an interactive three-dimensional
world in which the objects have a sense of spatial presence;
virtual environment, and virtual world are synonyms for virtual reality.
(Department of Defense, 2018)
\end{quotation}

The age of the first definition is apparent, but whether accepting older or newer
definition, the majority of them define VR as a kind of
virtual world that can be viewed, explored, or experienced by the user.


Virtual reality is sometimes mistakenly confused with Augmented Reality (AR).
For unenlightened people, these terms (VR and AR) can be hard to distinguish,
not to mention the other terms used in the industry, such as MR and XR
(mixed reality and extended reality, respectively). There is a very straightforward
way to differentiate VR from AR — VR is known to replace elements
(visual, sound) to provide full immersion in a completely different virtual
world, whereas AR often adds elements to the real world \hyperlink{vrar}{}.
The clearest example would be to compare the \emph{HTC Vive} VR headset with
\emph{Microsoft HoloLens} AR headset — Vive blocks the view entirely, putting
a screen in front of the eyes, while HoloLens is equipped with a holographic
display, which is transparent and allows to see through.


It is also essential to acknowledge, that virtual reality does not always mean
putting a headset on the head and controllers in the heads, as most of the
contemporary hardware users would expect. Especially in
early VR experiments, the visuals could be provided by projecting the images
on four walls surrounding the user (as in popular CAVE system \hyperlink{cave}{}) and
using surround sound for the spatial audio.


\section{Contemporary hardware}
The days of low-resolution, high-latency displays and expensive, unavailable,
and unprecise hardware sensors are long gone. Nowadays, VR headsets are
sold as consumer products and are available to almost everyone. They are
equipped with high-frequency refresh-rate displays, fast and low-latency
gyroscopes, and other high-end electronics components invented in the last
decades.


Current consumer VR headsets features, among other components,
a high resolution displays with refresh speeds often reaching 90Hz, gyroscopes
with up to 0.01\degree rotation precision, and stereo headphones with 3D sound.
Location tracking of the VR system can reach up to sub-millimetric precision
\hyperlink{vivenasa}{}, and motion-to-photon latency
\footnote{Latency measured between headset motion and photons emitted from the display.}
on some headsets range between 2-7 milliseconds. \hyperlink{mtpltc}{} \hyperlink{xinwiki}{}
VR headsets offer such precise tracking abilities, that it is suitable even
for research purposes and scientific projects \hyperlink{vivepbsr}{}, and therefore
it should not be surprising that VR is a popular topic to research.


\begin{figure}[h]{}
\centering\includegraphics[width=\textwidth]{assets/25688530252_e56eee6e9d_b.jpg}
\caption{}

\end{figure}

\section{Uses of virtual reality}
The investments made to AR and VR technologies suggest
that the most popular application of AR/VR is in the entertainment industry,
followed
by training situations, industrial maintenance and retail showcases.
\hyperlink{statistavr}{}


Advancements in haptic feedback allow for better surgery training using
VR simulations \hyperlink{vhfcrmisvrt}{}. Though no exact data proving performance gains
were not published yet, it can be expected that thanks to the correct haptic
feedback, VR training might become superior, compared to traditional
training methods. \hyperlink{vrsrgr}{}


\hypertarget{x-developing-for-virtual-reality}{\section{Developing for virtual reality}}
The contemporary state of virtual reality development tools is undoubtfully
coupled to the videogame industry, together with the procedures and practices.
It should not be surprising that the most popular tools for creating
VR applications are game engines and frameworks.


The current most notable popular game engines with the support of creating VR
applications are: Unity \footnote{\href{https://unity3d.com}{https://unity3d.com}} by
Unity Technologies, Unreal Engine \footnote{\href{https://unrealengine.com}{https://unrealengine.com}} by
Epic Games, or Godot Engine, the open-source engine developed by the community
under the MIT license. \footnote{\href{https://godotengine.org}{https://godotengine.org}} \hyperlink{slantvr}{}
