\chapter{The current state of haptic feedback in Virtual Reality}

Virtual reality is being researched and experimented with since the early 1960s.
\hyperlink{vrhistory}{} Since then, VR underwent many changes in terms of hardware,
software, and public popularity.
Some of the experiments were transformed into real commercial products.
Most of them are still in the early development phase, but they are slowly getting
closer to become finished consumer products.


Primary sources of haptic feedback, in current consumer virtual reality systems,
come from visual and sound elements produced by the VR headset displays and
headphones. Tracked
motion controllers provide vibration feedback and tactile buttons (many of
the controllers feature a "trigger" button, that can be squeezed and acts as
a real trigger, users intuitively feel when it is pressed or not).
Apart from the mentioned sources of feedback, no other sources can be found
within typical current VR systems.


\hypertarget{x-experiments-in-haptic-feedback}{\section{Experiments in haptic feedback}}
Some experiments try to provide haptic feedback on a full body with various
kinds of equipment, for instance, in the form of a wearable vest \hyperlink{tfbhf}{},
others focus the haptic feedback more on specific parts of the
human body, most often hands. There are multiple haptic gloves projects
\hyperlink{haptgloves}{}, using different approaches, but with the same goal — to allow user feel virtual objects with his hands
(touch them, pick or squeeze them). Some researches are tied to VR,
some of the projects mentioned in this chapter are not directly relevant
to VR, but can be theoretically used for enhancing it.


If we think about the characteristic properties of these haptic feedback
accessories, it seems that most of them are attached directly to the
user (hand gloves, hand controllers, head attached devices, or full-body wearables)
or interacting with him/her directly in some way. There is an opportunity to explore
the "external" feedback accessories, that affects the environment, rather than
the user directly.



The latest research in the field of hand haptics comes from Carneige Mellon
University in USA. With their project \textbf{Wireality}, they chose a different
approach and instead of gloves, they put a set of wires connected from the
accessory to the user’s fingers. The wires are moving freely and can be locked
in a position when needed. This solution claims to be non-expensive and
have low power consumption. \hyperlink{wireality}{}


\begin{figure}[h]{}
\centering\includegraphics[width=2.5truein]{assets/wireality.png}
\caption{}

\end{figure}

In 2016, the research staff of the Imagineering Institute of Malaysia
had presented a project called "Digital Smell Interface". In their book
"Virtual Taste and Smell Technologies for Multisensory Internet and Virtual
Reality" \hyperlink{vrstmivr}{}, they described experiments with electrical stimulation
of the inner nasal glands and by applying small currents of different
frequencies, they were able to simulate various smells.


A project called "Vocktail" is promising to simulate various tastes of a single
drink.
(Nimesha Ranasinghe, et al., 2017. Vocktail: A Virtual Cocktail for Pairing
Digital Taste, Smell, and Color Sensations) \hyperlink{vocktail}{} Vocktail — an electronic
device in the shape of a drinking glass, simulate various drink tastes,
which can be set using the accompanying mobile application. The device
generates high-frequency electric pulses at low currents. Paired with
different colors of the drinking glass, users are convinced they are drinking
liquids with different tastes.


The mentioned projects undoubtfully seem to be very unusual
(some might even say "scary") in the current state, but their looks and convenience
might be improved in the future. Currently, the devices are definitely not
suitable for VR, but relevant works might be something to
watch closely in the future.


\hypertarget{x-commercial-products-providing-haptic-feedback}{\section{Commercial products providing haptic feedback}}
Haptic feedback accessories do not have to end in the experimental state.
Many companies started to develop commercial products providing haptic feedback.


"Haptic suits" (also known as "tactile suits") are wearable
equipment, that can be paired with current virtual reality systems, providing
haptic feedback on the whole body.


One of the examples is \textbf{Teslasuit} — wearable suit for AR/VR that can provide
haptic feedback, precise motion capture, and various biometrics sensors to
enhance entertainment content or collect data for monitoring of
VR-enhanced training. \hyperlink{teslasuitab}{}


\begin{figure}[h]{}
\centering\includegraphics[width=2.5truein]{assets/TESLASUIT_Presentation.jpg}
\caption{}

\end{figure}

Haptic suits can simulate touch on various parts of the body, or simulate
vibrations synchronized with events happening in the virtual world. Some
of them provide thermal feedback to communicate virtual environment
conditions to the user.


It is expected that the commercial solutions for whole-body haptic feedback
will be expensive. Given the actual costs for the virtual reality setup
(typical user needs to buy VR headset and powerful graphics card for his computer
to use with VR), it is questionable if such suits will be
accessible for use in typical VR setups at home.



"Haptic gloves" also belong to the wearable hardware category. They are meant to be
used as a replacement for current VR controllers, and will provide not only
complex hand pose tracking, but most importantly will provide
haptic feedback for hands.


An example of haptic gloves product would be \textbf{HaptX} — a haptic
glove, that can provide touch haptic feedback and precise
hand motion tracking. Users wearing the glove can feel virtual structures
and patterns on their fingers, and the glove’s exoskeleton can limit user
hand movements, to simulate various surfaces and solid objects that the user can
touch or squeeze. \hyperlink{haptxtech}{}


\begin{figure}[h]{}
\centering\includegraphics[width=2.5truein]{assets/HaptX-Gloves-User.jpg}
\caption{}

\end{figure}


From a bit different corner of haptics comes the \textbf{Feelreal} multisensory
mask, made by company Feelreal Inc. This mask promise users to enhance
their VR experiences by simulating various smells by heating the perfume
cartridges right next to their noses (the mask is attached to the user’s face).
It stimulates another of the user’s senses and makes their
experience much more immersive. Apart from perfume release, the device is
also equipped with heaters and small fans that make users feel a warm
wind on their faces. One of the disadvantages of the device is the fact that
it only covers the lower part of the user’s head; therefore, the user can feel
the warm wind only on his cheeks and mouth, which seems to be limiting.