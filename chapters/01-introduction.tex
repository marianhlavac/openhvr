\chapter{Introduction}

\vspace{6em}

As of 2020, it is becoming quite rare to know someone who has not yet heard
of virtual reality. Virtual reality has found its way amidst the general public,
and nowadays, it feels like nearly everyone interested in technology,
electronics, or computers, have tried or at least seen someone using a virtual
reality headset.

It has become a recent topic grabbing attention in many industries — e.g.,
videogame industry, retailers, or professional training. The popularity
of virtual reality systems is on the rise \hyperlink{shipments}{}, and the size
of the virtual reality market is forecasted to reach 18.8 billion U.S. dollars
by 2020 \hyperlink{vrmsize}{}.

The conventional virtual reality system consists of a headset and controllers,
which provide users with visual and audio stimuli. Regarding haptic feedback,
typically, only controllers are used to produce vibration haptics. Without
additional accessories, virtual worlds cannot communicate any additional
information to virtual reality users.

This work sets its objective to utilize modern web technologies and IoT
devices to bring more types of haptic feedback to virtual reality users.
An easily extensible and open-sourced system will allow developers to design
VR experiences with external environmental effects, that will increase
immersion levels.

The idea for this work originates in the Dolby Atmos system — surround sound
technology developed by Dolby Laboratories, used in cinemas to provide
immersive and versatile 3D sound for cinematography. Instead of typical
multiple-channel audio, the producers of movies compose audio tracks with
3D object descriptors, and the resulting sound is mixed after the distribution
at the cinemas, to audio channels, that fit the speaker configuration
of the cinema. \hyperlink{dawp}{}

This work will use a similar concept, not only with sound but with various
external effects, such as wind or heat. A computer program will be given
the room configuration — positions of appliances able to create such effects,
and a 3D position of desired effect performance representing a condition
in the virtual world. The program then will decide which appliances should
be powered, which will make the virtual worlds more believable and immersive.

To denote the system created in this work in the text further, a codename
"OpenHVR" was chosen as the name of this project. More information about
the name in Chapter 7.

