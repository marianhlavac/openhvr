\chapter{Conclusion}

This work has concentrated on enhancing virtual reality experiences with 
additional external haptic feedback. Utilizing modern web technologies and 
IoT devices, a system has been built that works with any virtual reality 
equipment.

The first three chapters introduced the idea and presented the current state 
of current virtual reality technologies and the inventions in the sensory 
feedback field, whether experimental or commercial. The chapters mention 
several research papers, as well as notable commercial products that are 
currently in development.

The next chapter analyzed possibilities of a variety of effects that might 
be generated by such a system and evaluated which of them are suitable 
for use with virtual reality systems. It was achieved by evaluating each 
of the human senses by comfort, safety, availability of device able to affect 
the particular sense, and the device price. As a result, five senses 
and corresponding electrical appliances were determined to be suitable.

Chapters six and seven described the process of design and implementation 
details. The project was given the name `OpenHVR' and consists of three 
components — communication server, configurator tool, and plug-in for Unity 
Engine. The result incorporates not only the core implementation of 
the system but also the developers' tools in the form of the plug-in.

Finally, chapters eight and nine detailed the example application created 
for the implementation validation, which also helped to conduct the user 
testing. The user testing scenario contained three tests; two of them measured 
the properties of appliances used, the third one inquired on the user's opinions
on the work results. One of the tests produced unsatisfying results 
and therefore lacked a conclusion. Tester's opinions were, on average, 
quite positive and their suggestions were projected into the 
further plans of the project.

Hopefully, the project will continue with potential follow-up work of a 
fully-featured application (or game) that will practically use the new 
possibilities of VR experiences that this work might have unfolded.