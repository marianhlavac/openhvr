\chapter{Implementation}\label{implementation}

This chapter describes the process of implementing the OpenHVR system. \\
As chapter~\ref{analysis} stated, the system consists of three components: the
Communication Server, Configuration UI, and Unity Plug-in. First, the process 
of communication server implementation is described,
followed by the same description of the configurator and the plug-in. 
Lastly, the choice of the hardware used for testing the system is listed.


\hypertarget{x-communication-server}{\section{Communication Server}}
For the implementation of the server, the
\textbf{Go}\footnote{The Go Programming Language \href{http://golang.org}{http://golang.org}} language was
chosen. Among experienced developers, the design of the language is not 
considered as modern,\,\cite{gogbu} but on the other hand, it is
suited for writing performant server web applications, and easy
to learn and write the code. 

For the rapid development of the REST API, the \textbf{Beego}\footnote{\href{https://beego.me}{https://beego.me}}
framework was picked. It offers MVC architecture, URL routing abilities, ORM,
and other modules for the easy development of a webserver.


For persisting the data, \textbf{SQLite}\footnote{\href{https://www.sqlite.org/}{https://www.sqlite.org/}}
is used by the server as a database engine. SQLite 3 is
a popular database, and it does not require any complicated installation
(as opposed to, for example, PostgreSQL or MySQL) and can save all data into
memory or a single file. SQLite is suitable especially for modest data loads, as
it is not as performant as the mentioned alternatives. The server uses the
filesystem persistence of data. Conveniently, the Beego framework has built-in
ORM support for SQLite databases.

For unit testing, Convey\footnote{\href{https://github.com/smartystreets/goconvey}{https://github.com/smartystreets/goconvey}}
library is used. Unit tests cover the functions in models, helper functions,
and others.

\subsection{Device drivers}
To support the future extendability of the communication server, the program
will support multiple ``drivers'' -- pieces of code that receives a device information 
and effect request. The driver's purpose is to implement a reaction to the 
forwarded effect request and perform device-specific steps for controlling
various appliances.
This concept allows developers to write request handlers to extend the server’s 
abilities further and support larger range of devices compatible with OpenHVR.

The following diagram pictures the data flow of the incoming effect request.

\begin{figure}[h]{}
\centering\includegraphics[width=\textwidth]{assets/drivers-diagram.pdf}
\caption{Simplified program diagram of the device driver behavior when an effect request is received}
\end{figure}

Drivers will register at the initialization of the server (in the server code).
Each driver gets its name, which needs to be specified when adding a new
device using \texttt{POST /devices/}. For the clients, to know about available
drivers, a resource \texttt{GET /devices/drivers} returns a list
of currently supported drivers.


\hypertarget{x-configurator-tool}{\section{Configurator Tool}}
The assumptions when designing the configurator tool suggested that it will
be a web application. The application will allow configuring the system
on a computer or mobile device. Since the server provide REST API,
the configurator tool acts as a
web client (thin client) and uses this REST API to communicate and transfer
all data.


There is not much choice when picking a language for client-side applications
to run in current browsers. One of the currently
emerging candidates — \textbf{WebAssembly}\footnote{\href{https://webassembly.org}{https://webassembly.org}} — binary instruction format for web browsers, can be considered as
a present-day alternative to JavaScript. Even though WebAssembly is
currently available in almost all modern browsers,\,\cite{wasmroadmap} frameworks
and tools are rather new and many are still in early experimental state.


For the implementation, \textbf{JavaScript} language was picked, because
of the existing mature tools and frameworks. It is known that WebAssembly
is much faster than Javascript,\,\cite{wasmfast} but given the nature of the
application, it is fairly safe to say that using JavaScript, the application
will be fast enough to satisfy the \hyperref[cfg-n1]{CFG-N1}
(\emph{User interface must be fast and responsive})
requirement.

To build the user interface quickly, UI framework was used.
\textbf{Svelte}\footnote{\href{https://svelte.dev}{https://svelte.dev}} is a modern web framework for JavaScript
that can help build the configurator UI much faster and make the application
more reliable. Unlike other UI frameworks for JavaScript, Svelte offers
faster DOM updates\footnote{Document Object Model (DOM) is a programming interface for HTML and XML documents.\,\cite{dom}}
, transfer less unnecessary data to clients, and the framework code is run 
at build time rather than at runtime on client’s browsers.\,\cite{svelteblog} 
On the other side, Svelte does not officially
support transpilation\footnote{Transpiler is a source code translator, that does 
not translate code into bytecode or assembly (as typical compilers do), but 
translates code to different source code of the same or different language 
(for example TypeScript -> JavaScript)\,\cite{sscd}}
from other languages, so for example, it is not possible
to use TypeScript to write Svelte application with strongly typed code.


\subsection{Note on input with tracked controllers}
For the satisfaction of the requirement \hyperref[cfg-f4]{CFG-F4}, the process of the system configuration
ended up being split between two parts of the OpenHVR system — the central part is the front-end web application described
in this chapter. Additionally, a helper tool written in Unity has been added,
which is now part of the Unity Plug-in. This tool is used for determining
the precise location and rotation of effect devices in the VR space
coordinate frame, as this is not possible from the web browser (at least
not currently).


The usage of the web application configurator tool is still justified.
Configurator Tool can be opened and used while other VR application is currently
running (VR systems can run only one VR application), and offer current state
monitoring on, for example, a mobile device. Regarding UX, it would be uncomfortable to
input text or fill in complicated forms in VR. It is generally better to use web forms to
input configuration of the device (such as IP addresses or device name).


The helper tool acts as another client. It is using the REST API of the server
the same way as the front-end application. More details about the helper tool
can be found in the following chapter.


\section{Unity Plug-in}
Game scripts are used to extend the behavior of Unity applications. Even though
we call this component a ``plug-in'', in reality, it consists just of a
set of scripts and related assets, that can be imported into an existing
or new Unity project and used.


For scripting, the Unity Engine gives a choice between three languages — C\#, \\
UnityScript\footnote{UnityScript is a special variant of JavaScript}, and
Boo. From these three, the first one is the most popular and the most
feature-rich with the most complex binding to the Unity API.\,\cite{unityblog}
Because of those reasons, the prompt choice of language was C\#.


Through C\# scripts, Unity offers developers exceptional abilities to extend
the UI of the Unity editor. The plug-in leverages this advantage, as it can
create an easy to use UI for developers to define and launch effects.


\subsection{Gizmos}
Gizmos are editor icons used in Unity Editor and they are used for
visual aid and visual debugging.\,\cite{gizmos}
With these icons, developers can see the game objects, that are
typically not visible in the scene.


Plug-in uses Gizmos to visualize positions and directions of the
effects, and positions of the configured effect devices, when running the game.


For the OpenHVR plug-in, a set of icons were designed and are used for
Unity prefabs, that developers might use when using the plug-in. The icons
are designed to be similar to Unity’s default ones and therefore provide
a familiar user interface, partially satisfying the 
\hyperref[plg-n1]{PLG-N1} requirement.


\begin{figure}[h]{}
    \vspace{1em}
\centering\includegraphics[width=0.95\textwidth]{assets/icon-series.png}
\caption{Designed series of Gizmos icons for use in Unity editor}

\end{figure}

\hypertarget{x-location-helper-script}{\subsection{Location helper script}}
For easy and precise determination of effect devices locations and rotations
relative to the VR coordinate frame, a Unity helper script was created.


This script can be temporarily included in any Unity project or can be run
standalone in an empty project. It offers a simple interface:
User runs the application, picks one of the already registered OpenHVR
effect devices, and updates its position by moving the controller physically
to the place of the effect device and pressing the trigger.


By using this tool, each time the room layout changes, the configuration can
be updated by visiting all devices in the room and pressing the button at
correct locations.


\hypertarget{x-implementation-notes}{\section{Implementation notes}}
\hypertarget{x-versioning}{\subsection{Versioning}}
For versioning of the source code, Git\footnote{Git is a distributed version 
control system \href{https://git-scm.com}{https://git-scm.com}}
was used, together with Git LFS for
storing larger assets (such as textures, models, and pictures)
, for the implementation of the example application.


\hypertarget{x-docker-support}{\subsection{Docker support}}
Built binaries and configuration setup are packaged into Docker\footnote{Docker
 is container platform, using OS-level virtualization to deliver programs in 
 packages called containers \href{https://docker.com}{https://docker.com}}
image, that can be easily and quickly run on any machine.


Users are given a choice to compile the server manually, or if their machine has
Docker installed, they can download the image and run it without
the necessity of configuring the Go Tools to compile the source code.

More information on how to install with Docker is present in the Install Guide.

\hypertarget{x-hardware-used}{\section{Hardware used}}
In this chapter, specific hardware selection, which will be used for testing the
system’s implementation, is presented.

\subsection{ESP8266}

ESP8266 is a low-cost microchip with TCP/IP stack and WiFi connection, 
often used for rapid prototyping of IoT devices. It features
a 32-bit 80MHz microprocessor with 32 KiB of instruction memory and 80 KiB for
user data, and with up to 16 MiB of a flash storage.\,\cite{espspecs}

The integrated Wi-Fi module makes this chip suited for various IoT use-cases 
and is used in many current commercial IoT products or development boards.
Some of them are described in the following sections of this chapter.

\hypertarget{x-esp-01s-relay-boards}{\subsection{ESP-01S relay boards}}
One of the cheapest variants to make electronic appliance controllable
remotely is connecting them via ESP-01S relay boards with ESP8266 chips.
These boards can be bought very cheaply at popular online marketplaces
(depending on the seller, around US\$3), making it perfect for buying in
higher amounts to control many devices around the VR play-space in the room.

The main disadvantage of these cheap boards is their quality. In most cases,
they are not certified, and their parameters often cannot be trusted. Therefore,
these are suitable only for lower loads (like pedestal fans). Connecting high
loads (infrared heaters) might not be safe.


These boards come with plain firmware flashed into the memory. Alternative
firmware called `Tasmota' can be easily flashed using FTDI into the memory
of the chip. The advantages of the alternative firmware are described
in one of the following chapters.


\hypertarget{x-sonoff-smart-relays}{\subsection{Sonoff Smart Relays}}
When looking for a more safe and certified solution, while still keeping the
low-cost requirements, smart relays manufactured by company Sonoff
are a great choice. The model `Sonoff Basic' is certified for 10 A load,
theoretically allowing connection of appliances with power draw up to 2300 W (for the
electricity grid in our country). Other model, `Sonoff DUAL R2` features
two channels, which can be independently controlled, and is certified for 15 A
load.

\begin{figure}[h]{}
    \centering\includegraphics[width=0.8\textwidth]{assets/IMG_5236.jpeg}
    \caption{Sonoff DUAL R2 with two relay channels}
    \label{dualr2}
\end{figure}

\pagebreak

Most of the models of smart relays by Sonoff are based on ESP8266 chip;
because they can be flashed with the Tasmota firmware to provide non-proprietary
access to the device, they are very popular for consumers with interests to
flash custom firmware. With original firmware, the users are ``locked'' to use
Sonoff’s online cloud platform called `WeLink', to send and receive data.

For this work, the model `DUAL R2' was picked (fig.\,\ref{dualr2}). 
These relays are used to
control the infrared heater and some of the fans. DUAL R2 offers two output
channels and support for electrical load up to 15A total and can be powered
by voltages in the range 100-240V AC.

\hypertarget{x-tasmota — alternative-firmware-for-esp8266-based-devices}{\subsection{Tasmota — alternative firmware for ESP8266-based devices}}
Tasmota\footnote{\href{https://tasmota.github.io/docs/}{https://tasmota.github.io/docs/}} is an alternative open-source 
firmware for ESP8266-based devices.
As of April 2020, there are currently over 1180 devices supported,\cite{tasdirec}
which also includes many commercial consumer electronics based on ESP8266 chip,
that can be disassembled and ``hacked'' by flashing the alternative firmware
(such devices, unfortunately, will lose their warranty).
The firmware provides all necessary functions and non-proprietary
interfaces for communication over the TCP/IP, utilizing multiple protocols
(e.g., HTTP, MQTT).


The difficulty of flashing the firmware differs for each device. The programming pin
on the ESP8266 chip must be pulled down to the ground and connected to a computer using
any compatible FTDI device. There are many existing tools (e.g., 
esptool.py\footnote{\href{https://github.com/espressif/esptool}{https://github.com/espressif/esptool}})
that provides simple and easy to use command-line
interface for flashing a new firmware to the device. Additional information
on how to prepare a ESP8266 device are mentioned in the 
\nameref{installguide} on page \pageref{installguide}.

Devices equipped with Tasmota firmware can communicate over HTTP API or MQTT.

\hypertarget{x-results}{\section{Results}}
All three components of the system are implemented and specific hardware
for the testing environment was selected.


The communication server was created and provides live OpenAPI documentation.
A simple but sufficient web application for configuring the room for use with
OpenHVR was created. The web application is client-side and is included with
the OpenHVR server code. The server is hosting static files,
including the client-side application.
For using with Unity game engine, a set of scripts denoted as `Unity plug-in'
were also implemented. The devices communicate over HTTP API;
latencies of MQTT was unsatisfactory for the real-time communications, and Art-Net
did not meet the requirement for being a low-cost solution.

\begin{figure}[h]{}
\centering\includegraphics[width=\textwidth]{assets/running-server.png}
\caption{Screenshot of running server and the room configurator on the same machine}
\end{figure}

\begin{figure}[h]{}
\centering\includegraphics[width=\textwidth]{assets/configurator-ui.png}
\caption{Detailed screenshot of the resulting UI of room configurator}
\end{figure}

\begin{figure}[h]{}
\centering\includegraphics[width=\textwidth]{assets/openhvr-unity-plugin-usage.png}
\caption{Screenshot of Unity plug-in in use; the directional wind effect is set to simulate wind blowing from the virtual window.}
\end{figure}
