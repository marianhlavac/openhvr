\chapter{Example application using OpenHVR}

To demonstrate the abilities of the system, an example VR application
was created.


In the application, the user can walk around the virtual room and interact with
various items inside the room. Some of the items equip the OpenHVR
effects. Each example effect use is described below.


One of the user tests will use this example application, where
participants will be asked about an overall opinion on the experience and
how immersive it is.


The 3D models and materials were created using Microsoft Maquette
\footnote{VR prototyping tool, allowing a fast building of spatial environments \href{https://maquette.ms}{https://maquette.ms}}
and Unity’s XR Interaction Toolkit library provides the ability to
interact with various items inside the 3D worlds. This library allows users to
pick items with their hands and iteract with them intuitively (e.g., throwing them
or putting them somewhere).


The created models are very basic; there was not much of a focus on the design
of the models or making the scene photorealistic. Because of the limited time,
the scene design is functional and simplistic, not beautiful.


By making this example application, the OpenHVR project was validated and proved
to work as a whole and can be used for the desired purpose. The application uses
all parts of the system — the server must be running, the devices must be
connected and tracked their positions. The application is producing effects
using those devices, so the Unity plug-in must be used to produce the effects.


\begin{figure}[h]{}
\centering\includegraphics[width=\textwidth]{assets/test-room.png}
\caption{Overview of the example room scene}

\end{figure}

\hypertarget{x-testing-entities}{\section{Testing entities}}
The example application could be used for user testing. For those purposes,
the application scene includes two main testing entities.


\hypertarget{x-windows}{\subsection{Windows}}
There are multiple breakable windows in the room. Each of the interactable
windows can be set to produce various combinations of effects. Any combination
of the following effects can be turned on or off, according to testing
scenario requirements:


\begin{itemize}

\item Visual effect of blowing wind
\item Sound effect of blowing wind
\item Blue cold screen colorization to indicate changing temperature
\item OpenHVR wind effect at the location of the window

\end{itemize}


The room also contains interactable pieces of stones. Users can grab those
stones and throw them at the windows to break them. After breaking the window,
the effects will run.


\begin{figure}[h]{}
\centering\includegraphics[width=\textwidth]{assets/test-room-windows.png}
\caption{Screenshot of the windows inside the example scene}

\end{figure}

\hypertarget{x-fireplaces}{\subsection{Fireplaces}}
The virtual room will have multiple fireplaces. Users can grab a match and
start a fire. When a fire is started, similarily to the windows, an optional
combination of effects will start:


\begin{itemize}

\item Visual effect of fire

\item Sound effect of fire and cracking wood

\item Orange warm screen colorization to indicate the change of temperature

\item OpenHVR heat effect at the location of the fireplace

\end{itemize}


\begin{figure}[h]{}
\centering\includegraphics[width=\textwidth]{assets/test-room-fireplaces.png}
\caption{Screenshot of the fireplaces inside the example scene}

\end{figure}