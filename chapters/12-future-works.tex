\chapter{Future works}
The main appeal for follow-up work is to create an actual VR application
utilizing the OpenHVR project practically. The example application presented 
in this work is only for testing and presentation purposes and is not considered
as a full VR experience with enriched environmental effects. To present truly
immersive experience, the users expect excellent graphical quality of the 
environment.

It also might be interesting to think about configurable limitations specified
by the administrator.
Some effects might consume a considerable amount of electricity (especially
true for space heaters), and currently, developers designing the effects are
not limited in how many effects they can turn on at the same time and for how
long.

The failed execution of the fan direction test should be redone. The test
should give the desired results by paying attention to accurate tracking and
precise mapping real-world devices to virtual coordinates.

\section{WebXR}

WebXR capabilities could be incorporated into the Configurator Tool.
Users would not need to use the Unity Helper for Configurator to mark positions
of effect devices accurately. He/she would see the room
size and properties in the Web Configurator in real-time. Using a headset or
just the controllers, he/she would define the effect devices' positions and
rotations the same way as with the plug-in. 

On the other hand, the support in browsers for WebXR is still very experimental,
and more importantly, it is not guaranteed that the space coordinate frame
of WebXR applications will be the same as the space coordinate frame of OpenVR
or other runtime applications.

\pagebreak

\section{Service discovery}

Unity Plug-in should be able to discover the OpenHVR server automatically. 
This feature would mitigate the necessity to manually specify the IP address
of the computer running the OpenHVR server instance. Such a feature might
significantly enhance the user experience.