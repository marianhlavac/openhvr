\chapter{Proof of Concept results}

In April 2019, a proof of concept was performed as the coursework for the MI-MPR
course at FIT CTU\hyperlink{poc}{}. During the course, testing hardware was assembled,
and a simple Unity script was implemented.


The work shortly discussed the motivation and inspiration for the idea,
specified some goals, and presented a simple use-case of the system. In one of the
chapters, it defined some expected results and projected a work schedule.
Opportunities for various experiments were proposed.


Lastly, the implementation and results were described. The implementation
focused solely on wind simulation, as it is the most accessible and affordable
effect to simulate and perform tests on.


Some of the expected problems or experiment opportunities include, for example,
a need for taking in consideration the actuation time of the devices producing
the effects, or possibility of inability to reach the desired quality of effect
simulation, caused by having not enough fans around the user to simulate
full 360-directional wind effects.


The hardware consisted of two 3D-printed fan bodies, two DC toy motors and Arduino
UNO controlling the fans using PWM \footnote{Pulse Width Modulation}. Inside
the VR world, a red dot represented a wind source. Fan positions were manually
input. Unity plugin was constantly sending relative positions between the
wind source and fan positions to the Arduino board.


\begin{figure}[h]{}
\centering\includegraphics[width=2.5truein]{assets/IMG_1891.jpeg}
\caption{}

\end{figure}

For simplicity, the VR application was not room-scaled, but in the
user sitting/standing mode, and worked only on a 2-dimensional plane for
computing the direction differences between sources and effect devices.


The work successfully proved that such a system could be created and laid down
the foundations for transforming the idea into a master thesis project.

