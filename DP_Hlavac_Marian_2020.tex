% arara: xelatex
% arara: xelatex
% arara: xelatex
\documentclass[thesis=M,english,hidelinks]{template/FITthesis}[2019/12/23]
\usepackage[utf8]{inputenc} 
\usepackage[T1]{fontenc}

% \usepackage{subfig} %subfigures
% \usepackage{amsmath} %advanced maths
% \usepackage{amssymb} %additional math symbols
\usepackage{gensymb}
\usepackage{pdfpages}
\usepackage{dirtree} %directory tree visualisation

\DeclareUnicodeCharacter{2009}{ }
\DeclareUnicodeCharacter{200B}{ }
\linespread{1.2}
\setlength{\parindent}{0em}
\setlength{\parskip}{1.5em}

\setlrmarginsandblock{3.5cm}{3cm}{*}
\setulmarginsandblock{3.5cm}{*}{1}
\checkandfixthelayout 

% % list of acronyms
% \usepackage[acronym,nonumberlist,toc,numberedsection=autolabel]{glossaries}
% \iflanguage{czech}{\renewcommand*{\acronymname}{Seznam pou{\v z}it{\' y}ch zkratek}}{}
% \makeglossaries

% % % % % % % % % % % % % % % % % % % % % % % % % % % % % % 
% EDIT THIS
% % % % % % % % % % % % % % % % % % % % % % % % % % % % % % 

\department{Department of Software Engineering}
\title{Surrounding Environment Effects in Virtual Reality}
\authorGN{Marián} %author's given name/names
\authorFN{Hlaváč} %author's surname
\author{Marián Hlaváč} %author's name without academic degrees
\authorWithDegrees{Bc. Marián Hlaváč} %author's name with academic degrees
\supervisor{Ing. Jiří Chludil}
\acknowledgements{THANKS.}
\abstractEN{Summarize the contents and contribution of your work in a few sentences in English language.}
\abstractCS{V n{\v e}kolika v{\v e}t{\' a}ch shr{\v n}te obsah a p{\v r}{\' i}nos t{\' e}to pr{\' a}ce v {\v c}esk{\' e}m jazyce.}
\placeForDeclarationOfAuthenticity{Prague}
\keywordsCS{virtuální realita, internet of things, webové technologie, Unity Engine}
\keywordsEN{virtual reality, internet of things, web technologies, Unity Engine}
\declarationOfAuthenticityOption{1} %select as appropriate, according to the desired license (integer 1-6)
\website{https://github.com/mmajko/openhvr} %optional thesis URL


\begin{document}

% \newacronym{CVUT}{{\v C}VUT}{{\v C}esk{\' e} vysok{\' e} u{\v c}en{\' i} technick{\' e} v Praze}
% \newacronym{FIT}{FIT}{Fakulta informa{\v c}n{\' i}ch technologi{\' i}}


\setsecnumdepth{part}
\chapter{Introduction}\label{intro}

\vspace{8em}

As of 2020, it is becoming quite rare to know someone who has not yet heard
of virtual reality. Virtual reality has found its way amidst the general public,
and nowadays, it feels like nearly everyone interested in technology,
electronics, or computers, have tried or at least seen someone using a virtual
reality headset.

It has become a recent topic grabbing attention in many industries — e.g.,
videogame industry, retailers, education, or professional training. The popularity
of virtual reality systems is on the rise,\,\cite{shipments} and the size
of the virtual reality market is forecasted to reach 18.8 billion U.S. dollars
by 2020.\,\cite{vrmsize}

The conventional virtual reality system consists of a headset and controllers,
which provide users with visual and audio stimuli. Regarding haptic feedback,
typically, only controllers are used to produce vibration haptics. Without
additional accessories, virtual worlds cannot communicate any additional
information to virtual reality users.

This work sets its objective to utilize modern web technologies and IoT
devices to bring more types of haptic feedback to virtual reality users.
An easily extensible and open-sourced system will allow developers to design
VR experiences with external environmental effects, that will increase
immersion levels.

The idea for this work originates in the Dolby Atmos system — surround sound
technology developed by Dolby Laboratories, used in cinemas to create
immersive and adaptable 3D sound for cinematography. Instead of typical
multiple-channel audio, the producers of movies compose audio tracks with
3D object descriptors, and the resulting sound is mixed after the distribution
to cinemas into audio channels, that fit the speaker configuration
of the cinema.\,\cite{dawp}

This work will use a similar concept, not just with sound, but with various
external effects, such as wind or heat. A computer program will be given
the room configuration — positions of appliances able to produce such effects,
and a 3D position of desired effect performance representing a condition
in the virtual world. The program then will decide which appliances should
be powered, which will make the virtual worlds more believable and immersive.

To denote the system created in this work in the text further, a codename
`OpenHVR' was chosen as the name of this project. More information about
the name is mentioned in chapter~\ref{design} on page \pageref{design}.



\setsecnumdepth{all}
\chapter{Virtual reality}

Many definitions describe the term "virtual reality" (shortly VR) differently.
Some of them are older and define virtual reality as follows:

\begin{quotation}
Virtual reality is electronic simulation of environments experienced via
head mounted eye goggles and wired clothing enabling the end user to interact
in realistic three-dimensional situations (Coates, 1992)
\end{quotation}

Newer definitions define VR as:


\begin{quotation}
The use of computer technology to create an interactive three-dimensional
world in which the objects have a sense of spatial presence;
virtual environment, and virtual world are synonyms for virtual reality.
(Department of Defense, 2018)
\end{quotation}

The age of the first definition is apparent, but whether accepting older or newer
definition, the majority of them define VR as a kind of
virtual world that can be viewed, explored, or experienced by the user.


Virtual reality is sometimes mistakenly confused with Augmented Reality (AR).
For unenlightened people, these terms (VR and AR) can be hard to distinguish,
not to mention the other terms used in the industry, such as MR and XR
(mixed reality and extended reality, respectively). There is a very straightforward
way to differentiate VR from AR — VR is known to replace elements
(visual, sound) to provide full immersion in a completely different virtual
world, whereas AR often adds elements to the real world \hyperlink{vrar}{}.
The clearest example would be to compare the \emph{HTC Vive} VR headset with
\emph{Microsoft HoloLens} AR headset — Vive blocks the view entirely, putting
a screen in front of the eyes, while HoloLens is equipped with a holographic
display, which is transparent and allows to see through.


It is also essential to acknowledge, that virtual reality does not always mean
putting a headset on the head and controllers in the heads, as most of the
contemporary hardware users would expect. Especially in
early VR experiments, the visuals could be provided by projecting the images
on four walls surrounding the user (as in popular CAVE system \hyperlink{cave}{}) and
using surround sound for the spatial audio.


\section{Contemporary hardware}
The days of low-resolution, high-latency displays and expensive, unavailable,
and unprecise hardware sensors are long gone. Nowadays, VR headsets are
sold as consumer products and are available to almost everyone. They are
equipped with high-frequency refresh-rate displays, fast and low-latency
gyroscopes, and other high-end electronics components invented in the last
decades.


Current consumer VR headsets features, among other components,
a high resolution displays with refresh speeds often reaching 90Hz, gyroscopes
with up to 0.01\degree rotation precision, and stereo headphones with 3D sound.
Location tracking of the VR system can reach up to sub-millimetric precision
\hyperlink{vivenasa}{}, and motion-to-photon latency
\footnote{Latency measured between headset motion and photons emitted from the display.}
on some headsets range between 2-7 milliseconds. \hyperlink{mtpltc}{} \hyperlink{xinwiki}{}
VR headsets offer such precise tracking abilities, that it is suitable even
for research purposes and scientific projects \hyperlink{vivepbsr}{}, and therefore
it should not be surprising that VR is a popular topic to research.


\begin{figure}[h]{}
\centering\includegraphics[width=2.5truein]{assets/25688530252_e56eee6e9d_b.jpg}
\caption{}

\end{figure}

\section{Uses of virtual reality}
The investments made to AR and VR technologies suggest
that the most popular application of AR/VR is in the entertainment industry,
followed
by training situations, industrial maintenance and retail showcases.
\hyperlink{statistavr}{}


Advancements in haptic feedback allow for better surgery training using
VR simulations \hyperlink{vhfcrmisvrt}{}. Though no exact data proving performance gains
were not published yet, it can be expected that thanks to the correct haptic
feedback, VR training might become superior, compared to traditional
training methods. \hyperlink{vrsrgr}{}


\hypertarget{x-developing-for-virtual-reality}{\section{Developing for virtual reality}}
The contemporary state of virtual reality development tools is undoubtfully
coupled to the videogame industry, together with the procedures and practices.
It should not be surprising that the most popular tools for creating
VR applications are game engines and frameworks.


The current most notable popular game engines with the support of creating VR
applications are: Unity \footnote{\href{https://unity3d.com}{https://unity3d.com}} by
Unity Technologies, Unreal Engine \footnote{\href{https://unrealengine.com}{https://unrealengine.com}} by
Epic Games, or Godot Engine, the open-source engine developed by the community
under the MIT license. \footnote{\href{https://godotengine.org}{https://godotengine.org}} \hyperlink{slantvr}{}

\chapter{The current state of sensory feedback in virtual reality}

Virtual reality is being researched and experimented with since the early 1960s.
\cite{vrhistory} Since then, VR underwent many changes in terms of hardware,
software, and public prestige.
Some of the experiments were transformed into real commercial products.
Most of them are still in the early development phase, but they are slowly getting
closer to become finished consumer products.


Primary sources of sensory feedback, in current consumer virtual reality systems,
come from visual and sound elements produced by the VR headset displays and
headphones. Tracked
motion controllers provide vibration feedback and tactile buttons (many 
controllers feature a trigger button, that can be squeezed and acts as
a real trigger, users intuitively feel when it is pressed or not).
Apart from the mentioned sources of feedback, no other sources can be found
within typical current VR systems.


\section{Experiments in sensory feedback}
Some experiments try to provide haptic feedback on a full body with various
kinds of equipment, for instance, in the form of a wearable vest \cite{tfbhf},
others focus the haptic feedback more on specific parts of the
human body, most often hands. \cite{haptgloves} There are multiple haptic gloves 
projects, using different approaches, but with the same goal — 
to allow user feel virtual objects with his hands
(touch them, pick or squeeze them). Most researches are tied to VR,
but some of the projects mentioned in this chapter are not directly relevant
to VR, but can be theoretically used for enhancing it.

If we think about the characteristic properties of these haptic feedback
accessories, it seems that most of them are attached directly to the
user (hand gloves, hand controllers, head attached devices, or full-body wearables)
or interacting with him/her directly in some way. There is an opportunity to explore
the ``external'' feedback accessories, that affects the environment, rather than
the user directly.

The latest research in the field of hand haptics comes from the Carneige Mellon
University in USA. With their project \textbf{Wireality}, they chose a different
approach and instead of gloves, they put a set of wires connected from the
accessory to the user’s fingers. \cite{wireality} The wires are moving freely 
and can be locked in a position when needed. This solution claims to be 
non-expensive and have low power consumption. 


\begin{figure}[h]{}
\centering\includegraphics[width=0.8\textwidth]{assets/wireality.png}
\caption{The Wireality accessory in use and visualization of virtual touch (Photo courtesy of Carnegie Mellon University)}
\end{figure}

In 2016, the research staff of the Imagineering Institute of Malaysia
had presented a project called `Digital Smell Interface'. In their book
`Virtual Taste and Smell Technologies for Multisensory Internet and Virtual
Reality' \cite{vrstmivr}, they described experiments with electrical stimulation
of the inner nasal glands and by applying small currents of different
frequencies, they were able to simulate various smells.


A project called `Vocktail' is promising to simulate various tastes in a single
drink.
(Nimesha Ranasinghe, et al., 2017. Vocktail: A Virtual Cocktail for Pairing
Digital Taste, Smell, and Color Sensations) \cite{vocktail} Vocktail — an electronic
device in the shape of a drinking glass, simulate various drink tastes,
which can be set using the accompanying mobile application. The device
generates high-frequency electric pulses at low currents. Paired with
different colors of the drinking glass, users are convinced they are drinking
liquids with different tastes.


The mentioned projects undoubtfully seem to be very unusual in the current state
(some might be concerned about putting something on their nasal glands or run
eletrical current through their tongues), but their looks and convenience
might be improved in the future. Currently, the last two mentioned projects 
are definitely not suitable for VR, but relevant works might be something to
watch closely in the future.


\hypertarget{x-commercial-products-providing-haptic-feedback}{\section{Commercial products providing haptic feedback}}
Haptic feedback accessories do not have to end in the experimental state.
Many companies started to develop commercial products providing haptic feedback.


`Haptic suits' (also known as `tactile suits') are wearable
equipment, providing haptic feedback on the whole body. Haptic suits can 
simulate touch on various parts of the body, or simulate
vibrations synchronized with events happening in the virtual world. Some
of them provide thermal feedback to communicate virtual environment
conditions to the user.

One of the examples is \textbf{Teslasuit} — wearable suit that can bring
haptic feedback to AR/VR, together with precise motion capture, 
and various biometrics sensors to
enhance entertainment content or collect data for monitoring of
VR-enhanced training. \cite{teslasuitab}

\begin{figure}[h]{}
\centering\includegraphics[width=0.9\textwidth]{assets/TESLASUIT_Presentation.jpg}
\caption{Teslasuit, commercial haptic suit for virtual reality (Image courtesy of VR Electronics Ltd., Copyright © 2020 TESLASUIT)}
\end{figure}

It is expected that the commercial solutions for whole-body haptic feedback
will be expensive. Given the actual costs for the virtual reality setup
(typical user needs to buy VR headset and powerful graphics card for his computer
to use with VR), it is questionable if such suits will be
accessible for use in typical VR setups at home.

\pagebreak


`Haptic gloves' also belong to the wearable hardware category. They are meant to be
used as a replacement for current VR controllers, and will provide not only
complex hand pose tracking, but most importantly haptic feedback for hands.

\begin{figure}[h]{}
    \centering\includegraphics[width=0.8\textwidth]{assets/nasa-gloves.jpg}
    \caption{1980s wired haptic gloves used with virtual reality (Image courtesy of NASA)}
\end{figure}


An example of haptic gloves product would be \textbf{HaptX} — a haptic
glove, that can equip VR with touch haptic feedback and precise
hand motion tracking. Users wearing the glove can feel virtual structures
and patterns on their fingers, and the glove’s exoskeleton can limit user
hand movements, to simulate various surfaces and solid objects for users to
touch or squeeze. \cite{haptxtech}

From a different corner of sensory feedback comes \textbf{Feelreal} 
-- a multisensory mask made by company Feelreal Inc. This mask promise users to enhance
their VR experiences by simulating various smells by heating the perfume
cartridges right next to their noses (the mask is attached to the user’s face).
It stimulates another of the user’s senses and makes their
experience much more immersive. Apart from perfume release, the device is
also equipped with heaters and small fans that make users feel a warm
wind on their faces. One of the disadvantages of the device is the fact that
it only covers the lower part of the user’s head; therefore, the user can feel
the warm wind only on his cheeks and mouth, which seems to be limiting.
\input{chapters/04-existing-solutions}
\chapter{Analytical evaluation of the system}\label{analysis}

In this chapter, the system components and functionalities are being analyzed.
The functionality definition is preceded by the analysis of the human senses.
Each human sense is evaluated and decided if it is suitable to produce effects
that affect those senses. The results of this analysis are then used for
deciding what kind of effects the application will produce.


The system consists of the following components:


\begin{itemize}
  \itemsep0em
  \item Communication Server
  \item Configurator Tool
  \item Unity Plug-in
\end{itemize}


For each of the listed components, functional and non-functional requirements
were constructed. At the end of this chapter, various communication protocols
are assessed to help pick an appropriate way of communication between devices,
server, and the VR application.


\hypertarget{x-assessment-of-human-senses}{\section{Assessment of human senses}}
To correctly assess all possible suitable effects that can be produced, all 
externally affectable human senses have been listed, and each assessed,
if it is possible and appropriate to produce effects that can affect those senses.


The work is focused primarily on external effects. Inner senses, such as
hunger, thirst, or proprioception (sense of body motion), are neither
easily affectable by the standard electronics, but they also might not be safe
and comfortable for the users. The moral point of view
might also be in the way.

We are taking into consideration following human senses:
Visual~(sight), Auditory~(hearing), Somatosensory~(touch), Olfaction~(smell),
Gustation~(taste), and Equilibrioception~(balance). \\
Somatosensory sense can
be further divided into Mechanoreception~(skin~touch), Nociception (pain), and
Thermoception~(heat). Each of them is assessed more in detail further in this
chapter.


The process of affecting each sense has been evaluated by difficulty,
price, ergonomics, and safety. There might be various reasons why some
effects are not suitable for the use-case of this work.
Each of the senses is evaluated by categorizing into three types of results:


\begin{itemize}
  \itemsep0em

  \item \textbf{Suitable} — it might be appropriate to create effects affecting this sense,

  \item \textbf{Irrelevant} — even though it might be appropriate to affect this sense,
  there are reasons why it will not have any effect,

  \item \textbf{Not Suitable} — affecting such sense is inappropriate.

\end{itemize}


\hypertarget{x-irrelevant:-visual-(sight)}{\subsection{Irrelevant: Visual (sight)}}
Since all VR users are wearing a headset that completely blocks
all visual perceptions from the real world and ``replaces'' them by the visuals
created on the display of the VR headset, visual effects produced externally
will have no effect on the user.


Outside of the scope of this work, there might be an opportunity for experiments
to provide effects for people who are not using the VR headset and are
just watching. Such a situation can happen in VR arcades or simply
just between a group of people, where just one of them is wearing the VR headset.
By creating visual effects in sync with the VR
simulation, others can find watching someone else's experience
more enjoyable and entertaining. Nevertheless, for the time being and purposes
of this work, visual effects are considered as irrelevant.


\hypertarget{x-suitable:-auditory-(hearing)}{\subsection{Suitable: Auditory (hearing)}}
There might be the same argument as for the visuals — the majority of virtual reality headsets are already equipped with
headphones, and if not, users often use own headphones for
enhanced immersion in the virtual reality worlds.
Spatial audio\footnote{Spatial audio is a full sphere surround-sound technique that uses a 
dimensional approach to audio to mimic the way we hear in real life.\,\cite{spaudio}}
makes a real difference and is achieved usually using the headphones.
Surround sound is not typical for current virtual reality systems.

However, there is still one opportunity for enhancing the experience with sound.
It is particularly beneficial while using headphones. The typical frequency
response of headphones is from 20Hz to 20KHz,\,\cite{freqresp} but it is obvious
that lower ranges of frequencies produced by headphones are often much
less powerful and can not create large vibrations that can be felt by a whole
body. This type of effect partially belongs to somatosensory senses, because
lower frequencies can not only be heard but also felt. Creating vibrotactile
effects with bass speakers can simulate large thuds or explosions
happening within the virtual world.

\subsection{Suitable: Direct contact}

Simulating the touch is difficult. There are, for example, many receptors on
human fingers, which helps us feel various structures of solid
objects. We can touch the objects and feel the feedback. By squeezing the
object, we can deduce the object’s rigidity, and all this
information is very difficult to simulate and communicate to users from
virtual to the real world.

Some of the touch simulations are already possible, as explored in chapter~\ref{currentstate}, but all
of them are simulated by wearing various wearable equipment (gloves, suits),
or accessories attached to the user’s body.

Suitable are those machines that can produce effects externally (without an
all-time direct contact with the user), so they do not interfere with the VR
experience and can still create direct contact with the user when required.


\hypertarget{x-suitable:-feeling-of-moving-air}{\subsection{Suitable: Feeling of moving air}}
Humans are overall good at detecting winds and their direction. Our body hair
helps a lot. Moving air around using electronics is also very easy; fans
are specialized devices used primarily for the exact purpose.


Wind effects are very suitable and easily achievable as an effect that can
enhance virtual experiences. They are relatively safe, cheap to produce,
directional, and easily controllable.


\hypertarget{x-not-suitable:-feeling-of-wetness}{\subsection{Not Suitable: Feeling of wetness}}
Another way to cause skin sensations is to make the user feel the humidity.
The ability to disperse water in the room might drastically enhance immersion
in the virtual environment. Imagine standing next to the sea or in the rainy weather
inside of the VR scene.

Inspiration for this comes from these so-called `5D Cinemas' — a business
place (widely popular in shopping malls in the Czech Republic) offering
enhanced cinema experience. It uses a combination of 3D stereoscopic
pictures with various enhancing effects; one of them is a small water jet, 
which squirts small amounts of water to simulate a
situation in the scene of the movie, for example, a water splash.\,\cite{5dcin}


It is important to acknowledge that the environment in those 5D cinemas
is more controlled and ready for such conditions. The guests are not
wearing expensive equipment on their heads, and the equipment is
most probably water-proof, and therefore splashing water in the room will not
pose a risk of damaging the equipment.


In the case of a typical virtual reality setup, none of the equipment is ready
for contact with water or high humidity levels. For example — both the
\emph{Oculus Rift} virtual reality headset and the \emph{Oculus Touch} virtual reality
controllers have maximum operating humidity stated at 95\% RH (non-condensing).
\cite{orhswg}


\hypertarget{x-not-suitable:-nociception-(pain)}{\subsection{Not suitable: Nociception (pain)}}
Pain simulation made with electronic devices is possible. Electric currents can
induce pain of various amounts, depending on the amount of electric current
flowing through the body.
Equipment inducing pain electrically are available for consumers
(e.g., tasers, paralyzers, electric shock toys) and it is possible to
incorporate them to generate pain effects for VR simulation.


While pain simulation might greatly enhance virtual reality immersion
(especially in military simulators or computer games on war topics),
such devices also require direct contact with the user and, again,
it might affect user comfort when using the system and might get in between
when experiencing VR experience and affect it negatively.


However, the main concerns and reasons to discard such effect are the
safety and morality of such a device. Thus, such experiments will not be
a part of this work for obvious reasons, and are considered as not suitable.


\hypertarget{x-suitable:-thermoception-(heat)}{\subsection{Suitable: Thermoception (heat)}}
When kept in safe ranges, heat can help users to get more information about the
current scene in the virtual environment and distinguish between 
various ambient conditions.

The feeling of shining sun or heat coming from a cozy fireplace placed in the
virtual world can be simulated using heating elements placed in the real world.

The only risk relates particularly to safety measures, as heating elements are
a potential fire risk, and securing electronic devices producing heat effects
must be emphasized.


\hypertarget{x-suitable:-olfaction-(smell)}{\subsection{Suitable: Olfaction (smell)}}
To the current date, simulation of smell can be categorized as something unusual
or experimental. There are various attempts to simulate smell with electronic
devices; some projects are directly related to virtual reality technologies.


As researched in chapter~\ref{currentstate} and~\ref{relatedwork},
one approach to solving the problem is based on smell
cartridges that emit the smells by heating them with heating elements.
The main disadvantage of such a system is the need for maintenance — the
cartridges need to be replaced, which might turn up to be costly in the
long-term.


Concluding from the performed research, we still have too little
knowledge of how to simulate any smell precisely, or how to affect our organs
sensing smell. Currently, creating smell effects can be reliably achieved
only by heating perfume cartridges.


Although somewhat limited, such devices can be used for producing a simulation
of virtual world smells.


\hypertarget{x-not-suitable:-gustation-(taste)}{\subsection{Not Suitable: Gustation (taste)}}
Similarly to smell simulation, stimulating taste receptors electronically is
complicated as well. Although experiments can be considered more successful
compared to electronic smell simulation (mainly because of easier access
to taste receptors), it is still an early experiment.\,\cite{stsie}


Even if experiments were advanced and in a usable state, it would
require the user to have some kind of electronic device attached to the user’s
tongue. Such attachment might be uncomfortable for the user, especially when
used for long periods of time. Given the little potential of enhancing virtual
reality with taste, the negative effects will most probably balance out
the positive ones.


We can expect development in this field in the future. Suitable would be a product
that is safe and comfortable for long use and uses wireless
technology. But until such a product exists, working with taste simulation
in the current state is not suitable for the project.


\subsection{Not suitable: Equilibrioception (balance)}
To this date, we do not record any electronic device that could
directly affect body balance and simulate its state.

We know too little about controlling the body balance, and overall,
it might not be a good idea to affect the user’s balance. Losing
balance might result in users falling and damaging the equipment (headset and
controllers) or damaging the equipment in the room around the user.

VR is constantly fighting with user’s balance problems,
affecting balance perceptors could potentially be
counter-productive in efforts to eliminate motion sickness.

Affecting user balance is considered as not suitable.

\newpage


\hypertarget{x-overview}{\subsection{Overview}}
As a result of this assessment, a system for external effects for VR experience
enhancement can focus on four senses stimulation — hearing, touch, heat,
and smell.


For simplicity, this work will be focusing on just two of the mentioned suitable
effects — wind and heat.

\begin{table}[H]
\catcode`\-=12
\centering
\vspace{2em}
\begin{tabular}{|l|l|l|l|}
\hline
\multicolumn{3}{|l|}{\textbf{Sense}}                                                              & \textbf{Result}   \\ \hline
\multicolumn{3}{|l|}{Visual (sight)}                                                              & Irrelevant        \\ \hline
\multicolumn{3}{|l|}{Auditory (hearing)}                                                          & \textbf{Suitable} \\ \hline
\multirow{4}{*}{Somatosensory} & \multirow{3}{*}{Mechanoreception (skin touch)} & Direct contact  & \textbf{Suitable} \\ \cline{3-4} 
                                &                                                & Moving air      & \textbf{Suitable} \\ \cline{3-4} 
                                &                                                & Wetness, fluids & Not Suitable      \\ \cline{2-4} 
                                & \multicolumn{2}{l|}{Nociception (pain)}                          & Not Suitable      \\ \hline
\multicolumn{3}{|l|}{Thermoception (heat)}                                                        & \textbf{Suitable} \\ \hline
\multicolumn{3}{|l|}{Olfaction (smell)}                                                           & \textbf{Suitable} \\ \hline
\multicolumn{3}{|l|}{Gustation (taste)}                                                           & Not Suitable      \\ \hline
\multicolumn{3}{|l|}{Equilibrioception (balance)}                                                 & Not Suitable      \\ \hline
\end{tabular}
\caption{Overview table of assesment results}
\end{table}

\hypertarget{x-viable-electrical-appliances}{\section{Viable electrical appliances}}
Provided with the senses appropriate to affect, it is now important to determine
which electrical appliances can be used for creating effects that can trigger 
the mentioned senses.

We set categories of effects in the following table, and from now on will
refer to these effects by the category names. The choice of specific appliances
is discussed in chapter~\ref{implementation}.

  \begin{table}[H]
  \catcode`\-=12
  \centering
  \begin{tabular}{|l|l|l|}
  \hline
  \textbf{Sense}                          & \textbf{Category}                    & \textbf{Affectable by}      \\ \hline
  \multirow{3}{*}{Auditory (hearing)}     & \multirow{3}{*}{\textbf{Vibrations}} & Large speakers              \\ \cline{3-3} 
                                          &                                      & Subwoofer speakers          \\ \cline{3-3} 
                                          &                                      & Vibration generators/motors \\ \hline
  Mechanoreception (touch) & \textbf{Wind}                        & Pedestal fans               \\ \hline
  \multirow{2}{*}{Thermoception (heat)}   & \multirow{2}{*}{\textbf{Heat}}       & Heaters                     \\ \cline{3-3} 
                                          &                                      & Infrared heaters            \\ \hline
  Olfaction (smell)                       & \textbf{Smell}                       & Perfume dispensers          \\ \hline
  \end{tabular}
  \caption{Electrical appliances categories corresponding to senses}
  \end{table}

Appliance must be controllable programmatically over a computer network,
to act as a dynamic effect generator.

\pagebreak

There are appliances available on the market labeled as `smart'.
Briefly speaking, it means that the appliance
can connect to other devices wired or wirelessly for data exchange.\,\cite{wisd}
Such devices, in most cases, can send information they collect over
the network (e.g., weather stations collecting weather data, making them readable
on user’s smartphones), or able to listen to commands and perform various
kinds of actions (e.g., turn off a desk lamp, unlock door).

There are two ways to approach the selection of appliances. Either the
appliance can be smart and provide an interface of commands that can be sent, or
it can be a typical appliance connected via `smart wall sockets' — 
smart devices that can turn off the electrical power to appliances.

The main disadvantage of using a specialized smart device is the necessity of
working with different interfaces. There must be explicit
support in the server code for specific products. Devices created by different
manufacturers might behave differently.

The main disadvantage of the smart wall socket is the limitation in
control of the appliances. Fundamentally, the appliance can be either turned on or off.
This approach does not allow to precisely control the fan speeds or the power output
of heaters.

\section{Analysis of the appliances used}\label{analysis:appliances}
According to Table 4.2 and taking into consideration the
options available while working on the project, we will be using \textbf{fans
and infrared heaters}
to create wind currents and sources of heat, respectively.

Fans and infrared heaters will be controlled by a smart wall plug
and will be in two states — off and on. For each of the selected appliance
type, a set of properties will be defined
or measured and set as a `effect device properties' in the configuration
software. Such measurements will be taken as a part of the user testing.

\begin{table}[H]
\centering
\begin{tabular}{|p{6em}|p{20em}|p{6em}|}
\hline
\textbf{Property}        & \textbf{Description}                                                                         & \textbf{Expected values} \\ \hline
Actuation time           & The time the device needs to go from a turned-off state to a turned-on state.                & seconds                  \\ \hline
Directionality and range & The range and direction span of the area in which can be the effect experienced by the user. & seconds                  \\ \hline
\end{tabular}
\caption{Effect device properties}
\end{table}

The fact that the appliances differ by manufacturer, model, and type,
makes the measurements specific to each individual device.
For example, it is expected that the spin-up time
(actuation time) and range of wind effect produced by various pedestal fans
will be different, as such properties heavily depend on the power of the fan.

\pagebreak

For the testing environment built for this work, approximate measures will be
taken, and they will be provided in the configuration software as optional
recommended defaults. Administrators will be allowed
to measure their appliances by themselves and configure the properties with
their measured values.


The system presented in this work focuses more
on accessible hardware, open-source and non-proprietary solutions, and
the opportunity for more people to build their effect system in DIY style.


\section{Configurator tool analysis}\label{analysis:conftool}
Configurator Tool (alternatively `Room Configurator') is a web application that
can be used to input properties of the room and effect devices properties and
connection details.
The application should provide convenient GUI\footnote{Graphical User Interface}
for users to easily configure the system for their room and VR setup.


Through this application, users will define the location, rotation, type, and
additional configuration for each effect device placed in the room. The application
can also be used for various general configurations that might arise from
the implementation process, that could not be mentioned in the analysis.


\subsection{Functional requirements}
\paragraph*{User wants to configure his room for use with OpenHVR system <CFG-F1>}
Before using the OpenHVR system in a new space,
the room properties must be configured with the configurator tool.


\paragraph*{User wants to add an effect device into the room configuration <CFG-F2>}
Each device placed into a room that the user intends to use for producing the effects
must be connected with the communication server, and its type must be specified.
Additionally, more configuration might be required, depending on the type
of the device (e.g., thresholds, relay/channel selection, hardware limitations)


\paragraph*{User wants to define a location of added effect device <CFG-F3>}
Each device in a room that the administrator configured,
must have location and rotation (pose) information.


\paragraph*{User wants to input location information using one of the tracked controller in virtual reality space <CFG-F4>}
Additionally, to input location values, for user convenience,
the application will allow using a tracked controller to input the location at
various places for locating the effect
devices and mapping between the real world coordinate system and virtual
world coordinate system.


\subsection{Non-Functional requirements}

\paragraph*{User interface must be fast and responsive <CFG-N1>}
To provide satisfying user experience, the user interface should be fast and
responsive. The user interface should display loading progress and inform users
about currently ongoing actions.


\paragraph*{User interface must follow WCAG 2.1 <CFG-N2>}
The user interface must follow WCAG 2.1\footnote{Web Content Accessibility 
Guidelines (WCAG) 2.1 \href{https://www.w3.org/TR/WCAG21/}{https://www.w3.org/TR/WCAG21/}}
guidelines to provide an accessible user interface.


\section{Communication Server analysis}\label{analysis:server}

Communication Server is a web server, that acts as an intermediary component,
passing information between the IoT devices and the computer with
running VR application. This server holds data about room configuration,
status, and location of the effect devices and overall status of the system.

It must provide API\footnote{Application Programming Interface} to enable
information exchange between the computer running the simulation and IoT devices.


\subsection{Functional requirements}

\paragraph*{User wants to save a new configuration. <SRV-F1>}
\label{srv-f1}
After creating or editing a configuration in Configurator Tool, the user
want to save his changes and apply the effects.


\paragraph*{Unity plug-in will send information containing instructions for reproducing current scene effects <SRV-F2>}
\label{srv-f2}
Each effect happening inside the VR scene will be described as an effect
instruction. Using such instruction, the plug-in will create a request on the server to
reproduce described effects in the real world.


\paragraph*{Effect devices will expect instructions on how to behave <SRV-F3>}
\label{srv-f3}
All running effect devices will individually expect instructions for their
behavior. The server must receive instructions coming from Unity plug-in (in \hyperref[srv-f2]{SRV-F2})
and decide which devices will receive instructions and what content of the
instruction will be.
Practically speaking, if Unity plug-in asks to blow wind from the northern side of
the room, the server will determine which fans are located on the northern side
and send them instruction to start or stop spinning.

\pagebreak

\subsection{Non-Functional requirements}

\paragraph*{The server should be fast and responsive. <SRV-N1>}
\label{srv-n1}
Server will work with real-time data, and therefore, it should be fast and
responsive. The server should utilize asynchronous code (using threads or 
different methods) to be able handle multiple requests concurrently.

\paragraph*{The server should provide a standardized programming interface (RESTful API). <SRV-N2>}
\label{srv-n2}
To help developers easily integrate the server functionalities with other 
programs, the server will implement a RESTful API to provide more uniform
programming interface. The term `RESTful API' is described more in
Section 4.8.3.


\section{Unity Plug-in analysis}\label{analysis:plugin}

Current VR applications do not provide any standardized way
of gathering detailed environmental information about the simulation.
Most often, such details are not generally simulated by the application
(for example, not all VR applications simulate wind currents or
temperature in the scene).

For providing such information to the OpenHVR system, so that it can
reproduce virtual scene conditions in the real world, a custom Unity plug-in
will be implemented.

This plug-in will interoperate with Unity’s Transforms\footnote{Unity’s built-in components implementing the scene graph and defining location and rotation of the objects.}
to determine the effect location. Thanks to the
componential architecture of the game engine, the plug-in can offer
a component object, that can be attached to any game object.
Developers can then use the component in the same way as they use the other
components.

Plug-in components can be similar to existing Unity components, which
developers are used to. For example — the definition of vibration effect can be
very similar to defining a 3D audio source in Unity. In the same way, 
the developers will be able to fine-tune the effect type and its range.

\begin{figure}[H]{}
  \centering\includegraphics[width=\textwidth]{assets/unity-components.png}
  \caption{Screenshot of Unity componential architecture visible in the Unity Editor UI}
\end{figure}

\begin{figure}[H]{}
  \centering\includegraphics[width=\textwidth]{assets/unity-audio.png}
  \caption{3D Audio source set in a Unity Engine scene; the speaker icon defines 
  the location of the sound source, and the blue sphere around it defines the 
  range of the 3D sound}
  \end{figure}

\pagebreak

\subsection{Functional requirements}

\paragraph*{Developer wants to produce an effect at some location in the game world <PLG-F1>}
\label{plg-f1}
Using the component provided by the Unity plug-in, the developer will attach
a component to any object with a Transform component. The Transform component
will give the location of the effect in the game world. The developer will
set and trigger an effect by sending signals to the effect source component.

\paragraph*{Developer wants to use reference points of existing devices in the game world <PLG-F2>}
\label{plg-f2}
The developer can receive
positions and rotations of effect devices to produce effects in the
game world at better and more accurate positions.
This function will be used in the example app. Location points of
fans will be collected, and one of them will be picked and used to alter
the position of the virtual window (it will help to pick the correct wall position,
including the height of the window).

\subsection{Non-Functional requirements}

\paragraph*{Provided resources will be standardized among the Unity Engine environment <PLG-N1>}
\label{plg-n1}
For the implementation of the plug-in, native tools, UI elements, and properties
will be used to achieve creating an interface between Unity and OpenHVR.
Interface and tools should feel familiar for Unity developers.


\section{OpenHVR System analysis}

Users will often come into contact with the system as a whole. End-users
often cannot distinguish between the specific parts or components. From such
users, specific requirements arise. These requirements are not coupled with
any specific component, but rather imposes requirements on the system as a whole.


\subsection{Non-Functional requirements}

\paragraph*{User wants the effects not affect his virtual reality experience negatively <SYS-N1>}
\label{sys-n1}
OpenHVR should not affect the original virtual reality experience in any
negative way. For example, no such effect, produced by the system, should ever
constrain users from experiencing some parts of the original VR experience.

\paragraph*{OpenHVR should not put excessive pressure on system resources of the computer running VR applications <SYS-N2>}
\label{sys-n2}
Regarding system resources, VR applications are very demanding. It must be
made sure that OpenHVR will not use excessive amounts of system resources,
to keep the VR applications running smoothly.

\paragraph*{Reproduction of the technical setup should not be expensive and accessible <SYS-N3>}
\label{sys-n3}
OpenHVR is focused on being non-expensive. The basic technical setup should be
optimized on price, and consists of commonly available equipment.


\section{Means of communication analysis}

It was established that the effect devices would communicate over a computer network.
There are many network protocols, with different properties. The chapter will
analyze possible communication protocols and tools for communication
between smart devices, web servers, and VR applications.


\hypertarget{x-art-net}{\subsection{Art-Net}}
Firstly let us focus on a protocol that would seem to be the best for controlling
physical devices, such as lights or fans. Art-Net is a network protocol for
the distribution of data over an Ethernet network. It supports the connection of DMX
devices, which are most often used for stage lights. It uses a UDP-based packet
structure.\,\cite{artnet}

Art-Net is mostly used for lighting live performances. The first version,
`Art-Net I', was released in 1998. The latest 4th version released
in September 2016
is called `Art-Net 4'. Art-Net is hence matured and widely used in the
entertainment industry.

Equipment supporting Art-Net is professional-grade, 
built to be very reliable, which raises the price point
by a lot and generally makes them unavailable for a typical consumer.


\hypertarget{x-mqtt}{\subsection{MQTT}}
MQTT is a lightweight messaging protocol for small sensors and mobile devices,
optimized for high-latency or unreliable networks.  It is
based on a publish-subscribe concept, and the messages are sorted into a ``topics'',
devices can subscribe to messages published under a topic, or publish
a message into the topic.\cite{mqtthp}


Considering the fact, that the devices will generally be a smart home
electronics, MQTT might be a good choice, because many of them have built-in
support for MQTT or use MQTT as the primary protocol for communication.


The main disadvantage of MQTT is the unreliability in terms of latency.
It heavily depends on the implementation of MQTT on each of the device and on
the implementation of the MQTT server. The latency range is usually
in tens or hundreds of milliseconds for the most popular implementations,
\cite{mqttlat} which is sufficient, but the unpredictability is making MQTT
less suitable for use with real-time effects.

\pagebreak

\subsection{HTTP (REST)}
The most straightforward way of communication between devices would be to use
HTTP protocol, which is the most widespread protocol used in computer networks
and is used every day by billions of users.\,\cite{httpsrv}


A small disadvantage might be the large versatility of the HTTP protocol. There would be
a need for a standardized API for communication between devices and with the
server. To mitigate this disadvantage, a RESTful API can be designed to provide
the standardized API with expectable results.


REST (REpresentational State Transfer) is an architectural style for developing
web services and their interfaces. It defines constraints and conventions to
offer greater performance, scalability, simplicity, and more uniform interface.
\cite{restdef} ``RESTful'' is API that conforms to the REST architectural style.
\chapter{Proof of Concept results}

In April 2019, a proof of concept was performed as the coursework for the MI-MPR
course at FIT CTU. During the course, testing hardware was assembled,
and a simple Unity script was implemented.


The work shortly discussed the motivation and inspiration for the idea,
specified some goals, and presented a simple use-case of the system. In one of the
chapters, it defined some expected results and projected a work schedule.
Opportunities for various experiments were proposed.


Lastly, the implementation and results were described. The implementation
focused solely on wind simulation, as it is the most accessible and affordable
effect to simulate and perform tests on.


Some of the expected problems or experiment opportunities included, for example,
a need for taking in consideration the actuation time of the devices producing
the effects, or possibility of inability to reach the desired quality of effect
simulation, caused by having not enough fans around the user to simulate
full 360-directional wind effects.


The hardware consisted of two 3D-printed fan bodies, two DC toy motors and Arduino
UNO controlling the fans using PWM. Inside
the VR world, a red dot represented a wind source. Fan positions were manually
input. Unity script was constantly sending relative positions between the
wind source and fan positions to the Arduino board.


\begin{figure}[h]{}
\centering\includegraphics[width=\textwidth]{assets/IMG_1891.jpeg}
\caption{
    Proof of concept prototype setup; the red ball inside the Unity application in the background defines the wind direction.
}
\end{figure}

For simplicity, the VR application was not room-scaled, but in the
user sitting/standing mode, and worked only on a 2-dimensional plane for
computing the direction differences between sources and effect devices.


The work successfully proved that such a system could be created and laid down
the foundations for transforming the idea into a master thesis project. The subjective
feeling of the wind simulation was unsatisfactory and not effective, 
as a result of the wind sources being too small and not powerful. 
The technical setup was enough only for the technical validation.


\chapter{Designing the System}

The name "OpenHVR" was chosen as a codename and project name, under which
all source codes will be publicly available. The name emphasizes the open-source
essence of the project, and HVR stands for "Haptics in VR". A simple logo, to
go with the name, was also designed.


\begin{figure}[h]{}
\centering\includegraphics[width=0.5\textwidth]{assets/openhvrlogo.png}
\caption{}

\end{figure}

\hypertarget{x-components-cooperation}{\section{Components cooperation}}
The following simple diagram is picturing the system components and the
communication between them.


\begin{figure}[h]{}
\centering\includegraphics[width=\textwidth]{assets/openhvr-diagram.pdf}
\caption{}

\end{figure}


\hypertarget{x-configurator-tool-wireframe-prototype}{\section{Configurator Tool wireframe prototype}}
As a first step in designing the Configurator Tool, a wireframe prototype
has been designed. Configurator Tool user interface will consist of only
two interactive screens — the main screen and screen for adding a new device.
On the main screen, the user will be able to see an overview of the room configuration,
manage (move, update, remove) effect devices, and will see the system’s status.
The "New device screen" will be displayed after clicking on the "Add device"
button at the main screen.


The prototype pictures layout of the user interface.


\begin{figure}[h]{}
\centering\includegraphics[width=\textwidth]{assets/wireframe-1-dash.pdf}
\caption{}

\end{figure}

\begin{figure}[h]{}
\centering\includegraphics[width=\textwidth]{assets/wireframe-2-editing.pdf}
\caption{}

\end{figure}

\begin{figure}[h]{}
\centering\includegraphics[width=\textwidth]{assets/wireframe-3-adding.pdf}
\caption{}

\end{figure}

\hypertarget{x-rest-api-design}{\section{REST API Design}}
Following is a list of resources available at the server’s API endpoint.
With short descriptions it represents a simple draft of functions provided 
by the server component.
Concrete resource and schemas documentation derived
directly from the implementation, is present
in \hyperlink{./13-developer-guide#server-api}{} of the
Developer Guide chapter.


\begin{description}

\item[GET /devices/]Returns a list of registered effect devices in the current room configuration,
together with their properties (e.g., location, rotation, name).

\item[POST /devices/]Registers a new effect device in the room.

\item[GET /devices/drivers]Returns a list of currently available drivers. New drivers can be added only
by extending the source code of the server.

\item[GET /devices/{deviceId}]Returns details about a specific device.

\item[PUT /devices/{deviceId}]Updates specific device properties (e.g., location, rotation, name).

\item[DELETE /devices/{deviceId}]Removes registration of the device from the room configuration.

\item[POST /effects/]Requests a new effect performance. This resource is meant to be requested by the
client each time there is an event happening in the VR scene, which should also
produce an accompanying effect in the room. The parameters sent by the client
(i.e., type of the effect, duration, location and direction, range) are used
by the server to calculate which effect devices need to be instructed and sends
the instructions to them.

\item[DELETE /effects/]Immediately cancels all previously requested effect performances. Useful when
the scene needs to be cleared, or, for any reason, it needs to be reassured that
no devices are producing any effects (for example, when quitting the VR
application).

\item[DELETE /effects/{effectId}]Immediately cancels a specific, previously requested effect performance.
Canceling specific performances is needed only when infinite duration
was specified when requesting. Otherwise, the effect will end after the
duration, and there is no need to cancel it.

\item[GET /effects/types]Returns list of available types of effects. Client applications can request
this list at their startup and raise a warning at the client-side if VR
application is requesting an effect that is not available.

\item[GET /system/status]Returns system status.

\end{description}


\hypertarget{x-api-security}{\section{API Security}}
The server will not support authentication, and all resources will be publicly
available. It is expected that each server will serve a single instance of
a virtual reality set-up and will run on a local network without public access.
Should the user ever be concerned about unwanted access to the system, the server
will support IP whitelisting
\footnote{IP Whitelisting is a common name for the process of filtering requests by the sender IP address and letting through only the requests with senders listed on the "whitelist".}.


The security of each of the device depends on the software and firmware provided
with the smart device itself. Drivers included in the OpenHVR server should
support secure connection too. Tasmota firmware supports username and
password authentication.

\chapter{Implementation}\label{implementation}

This chapter describes the process of implementing the OpenHVR system. \\
As chapter~\ref{analysis} stated, the system consists of three components: the
Communication Server, Configuration UI, and Unity Plug-in. First, the process 
of communication server implementation is described,
followed by the same description of the configurator and the plug-in. 
Lastly, the choice of the hardware used for testing the system is listed.


\hypertarget{x-communication-server}{\section{Communication Server}}
For the implementation of the server, the
\textbf{Go}\footnote{The Go Programming Language \href{http://golang.org}{http://golang.org}} language was
chosen. Among experienced developers, the design of the language is not 
considered as modern,\,\cite{gogbu} but on the other hand, it is
suited for writing performant server web applications, and easy
to learn and write the code. 

For the rapid development of the REST API, the \textbf{Beego}\footnote{\href{https://beego.me}{https://beego.me}}
framework was picked. It offers MVC architecture, URL routing abilities, ORM,
and other modules for the easy development of a webserver.


For persisting the data, \textbf{SQLite}\footnote{\href{https://www.sqlite.org/}{https://www.sqlite.org/}}
is used by the server as a database engine. SQLite 3 is
a popular database, and it does not require any complicated installation
(as opposed to, for example, PostgreSQL or MySQL) and can save all data into
memory or a single file. SQLite is suitable especially for modest data loads, as
it is not as performant as the mentioned alternatives. The server uses the
filesystem persistence of data. Conveniently, the Beego framework has built-in
ORM support for SQLite databases.

For unit testing, Convey\footnote{\href{https://github.com/smartystreets/goconvey}{https://github.com/smartystreets/goconvey}}
library is used. Unit tests cover the functions in models, helper functions,
and others.

\subsection{Device drivers}
To support the future extendability of the communication server, the program
will support multiple ``drivers'' -- pieces of code that receives a device information 
and effect request. The driver's purpose is to implement a reaction to the 
forwarded effect request and perform device-specific steps for controlling
various appliances.
This concept allows developers to write request handlers to extend the server’s 
abilities further and support larger range of devices compatible with OpenHVR.

The following diagram pictures the data flow of the incoming effect request.

\begin{figure}[h]{}
\centering\includegraphics[width=\textwidth]{assets/drivers-diagram.pdf}
\caption{Simplified program diagram of the device driver behavior when an effect request is received}
\end{figure}

Drivers will register at the initialization of the server (in the server code).
Each driver gets its name, which needs to be specified when adding a new
device using \texttt{POST /devices/}. For the clients, to know about available
drivers, a resource \texttt{GET /devices/drivers} returns a list
of currently supported drivers.


\hypertarget{x-configurator-tool}{\section{Configurator Tool}}
The assumptions when designing the configurator tool suggested that it will
be a web application. The application will allow configuring the system
on a computer or mobile device. Since the server provide REST API,
the configurator tool acts as a
web client (thin client) and uses this REST API to communicate and transfer
all data.


There is not much choice when picking a language for client-side applications
to run in current browsers. One of the currently
emerging candidates — \textbf{WebAssembly}\footnote{\href{https://webassembly.org}{https://webassembly.org}} — binary instruction format for web browsers, can be considered as
a present-day alternative to JavaScript. Even though WebAssembly is
currently available in almost all modern browsers,\,\cite{wasmroadmap} frameworks
and tools are rather new and many are still in early experimental state.


For the implementation, \textbf{JavaScript} language was picked, because
of the existing mature tools and frameworks. It is known that WebAssembly
is much faster than Javascript,\,\cite{wasmfast} but given the nature of the
application, it is fairly safe to say that using JavaScript, the application
will be fast enough to satisfy the \hyperref[cfg-n1]{CFG-N1}
(\emph{User interface must be fast and responsive})
requirement.


To build the user interface more quickly, it was decided to use a UI framework.
\textbf{Svelte}\footnote{\href{https://svelte.dev}{https://svelte.dev}} is a modern web framework for JavaScript
that can help build the configurator UI much faster and make the application
more reliable. Unlike other UI frameworks for JavaScript, Svelte offers
faster DOM updates\footnote{Document Object Model (DOM) is a programming interface for HTML and XML documents.\,\cite{dom}}
, transfer less unnecessary data to clients, and the framework code is run 
at build time rather than at runtime on client’s browsers.\,\cite{svelteblog} 
On the other side, Svelte does not officially
support transpilation\footnote{Transpiler is a source code translator, that does 
not translate code into bytecode or assembly (as typical compilers do), but 
translates code to different source code of the same or different language 
(for example TypeScript -> JavaScript)\,\cite{sscd}}
from other languages, so for example, it is not possible
to use TypeScript to write Svelte application with strongly typed code.


\subsection{Note on input with tracked controllers}
For the satisfaction of the requirement \hyperref[cfg-f4]{CFG-F4}, the process of the system configuration
ended up being split between two parts of the OpenHVR system — the central part is the front-end web application described
in this chapter. Additionally, a helper tool written in Unity has been added,
which is now part of the Unity Plug-in. This tool is used for determining
the precise location and rotation of effect devices in the VR space
coordinate frame, as this is not possible from the web browser (at least
not currently).


The usage of the web application configurator tool is still justified.
Configurator Tool can be opened and used while other VR application is currently
running (VR systems can run only one VR application), and offer current state
monitoring on, for example, a mobile device. Regarding UX, it would be uncomfortable to
input text or fill in complicated forms in VR. It is generally better to use web forms to
input configuration of the device (such as IP addresses or device name).


The helper tool acts as another client. It is using the REST API of the server
the same way as the front-end application. More details about the helper tool
can be found in the following chapter.


\section{Unity Plug-in}
Game scripts are used to extend the behavior of Unity applications. Even though
we call this component a ``plug-in'', in reality, it consists just of a
set of scripts and related assets, that can be imported into an existing
or new Unity project and used.


For scripting, the Unity Engine gives a choice between three languages — C\#, 
UnityScript\footnote{UnityScript is a special variant of JavaScript}, and
Boo. From these three, the first one is the most popular and the most
feature-rich with the most complex binding to the Unity API.\,\cite{unityblog}
Because of those reasons, the prompt choice of language was C\#.


Through C\# scripts, Unity offers developers exceptional abilities to extend
the UI of the Unity editor. The plug-in leverages this advantage, as it can
create an easy to use UI for developers to define and launch effects.


\subsection{Gizmos}
Gizmos are editor icons used in Unity Editor and they are used for
visual aid and visual debugging.\,\cite{gizmos}
With these icons, developers can see the game objects, that are
typically not visible in the scene.


Plug-in uses Gizmos to visualize positions and directions of the
effects, and positions of the configured effect devices, when running the game.


For the OpenHVR plug-in, a set of icons were designed and are used for
Unity prefabs, that developers might use when using the plug-in. The icons
are designed to be similar to Unity’s default ones and therefore provide
a familiar user interface, partially satisfying the 
\hyperref[plg-n1]{PLG-N1} requirement.


\begin{figure}[h]{}
    \vspace{1em}
\centering\includegraphics[width=0.95\textwidth]{assets/icon-series.png}
\caption{Designed series of Gizmos icons for use in Unity editor}

\end{figure}

\hypertarget{x-location-helper-script}{\subsection{Location helper script}}
For easy and precise determination of effect devices locations and rotations
relative to the VR coordinate frame, a Unity helper script was created.


This script can be temporarily included in any Unity project or can be run
standalone in an empty project. It offers a simple interface:
User runs the application, picks one of the already registered OpenHVR
effect devices, and updates its position by moving the controller physically
to the place of the effect device and pressing the trigger.


By using this tool, each time the room layout changes, the configuration can
be updated by visiting all devices in the room and pressing the button at
correct locations.


\hypertarget{x-implementation-notes}{\section{Implementation notes}}
\hypertarget{x-versioning}{\subsection{Versioning}}
For versioning of the source code, Git\footnote{Git is a distributed version 
control system \href{https://git-scm.com}{https://git-scm.com}}
was used, together with Git LFS for
storing larger assets (such as textures, models, and pictures)
, for the implementation of the example application.


\hypertarget{x-docker-support}{\subsection{Docker support}}
Built binaries and configuration setup are packaged into Docker\footnote{Docker
 is container platform, using OS-level virtualization to deliver programs in 
 packages called containers \href{https://docker.com}{https://docker.com}}
images, that can be easily and quickly run on any machine.


Users are given a choice to compile the server manually, or if their machine has
Docker installed, they can download the images and run them without
the necessity of configuring the Go compiler and compiling it.

More information on how to install with Docker is present in the Install Guide.

\hypertarget{x-hardware-used}{\section{Hardware used}}
In this chapter, specific hardware selection, which will be used for testing the
system’s implementation, is presented.

\subsection{ESP8266}

ESP8266 is a low-cost microchip with TCP/IP stack and WiFi connection, 
often used for rapid prototyping of IoT devices. It features
a 32-bit 80MHz microprocessor with 32 KiB of instruction memory and 80 KiB for
user data, and with up to 16 MiB of a flash storage.\,\cite{espspecs}

The integrated Wi-Fi module makes this chip suited for various IoT use-cases 
and is used in many current commercial IoT products or development boards.
Some of them are described in the following sections of this chapter.

\hypertarget{x-esp-01s-relay-boards}{\subsection{ESP-01S relay boards}}
One of the cheapest variants to make electronic appliance controllable
remotely is connecting them via ESP-01S relay boards with ESP8266 chips.
These boards can be bought very cheaply at popular online marketplaces
(depending on the seller, around US\$3), making it perfect for buying in
higher amounts to control many devices around the VR play-space in the room.

The main disadvantage of these cheap boards is their quality. In most cases,
they are not certified, and their parameters often cannot be trusted. Therefore,
these are suitable only for lower loads (like pedestal fans). Connecting high
loads (infrared heaters) might not be safe.


These boards come with plain firmware flashed into the memory. Alternative
firmware called `Tasmota' can be easily flashed using FTDI into the memory
of the chip. The advantages of the alternative firmware are described
in one of the following chapters.


\hypertarget{x-sonoff-smart-relays}{\subsection{Sonoff Smart Relays}}
When looking for a more safe and certified solution, while still keeping the
low-cost requirements, smart relays manufactured by company Sonoff
are a great choice. The model `Sonoff Basic' is certified for 10 A load,
theoretically allowing connection of appliances with power draw up to 2300 W (for the
electricity grid in our country). Other model, `Sonoff DUAL R2` features
two channels, which can be independently controlled, and is certified for 15 A
load.

\begin{figure}[h]{}
    \centering\includegraphics[width=0.8\textwidth]{assets/IMG_5236.jpeg}
    \caption{Sonoff DUAL R2 with two relay channels}
    \label{dualr2}
\end{figure}

\pagebreak

Most of the models of smart relays by Sonoff are based on ESP8266 chip;
because they can be flashed with the Tasmota firmware to provide non-proprietary
access to the device, they are very popular for consumers with interests to
flash custom firmware. With original firmware, the users are ``locked'' to use
Sonoff’s online cloud platform called `WeLink', to send and receive data.

For this work, the model `DUAL R2' was picked (fig.\,\ref{dualr2}). 
These relays are used to
control the infrared heater and some of the fans. DUAL R2 offers two output
channels and support for electrical load up to 15A total and can be powered
by voltages in the range 100-240V AC.

\hypertarget{x-tasmota — alternative-firmware-for-esp8266-based-devices}{\subsection{Tasmota — alternative firmware for ESP8266-based devices}}
Tasmota\footnote{https://tasmota.github.io/docs/} is an alternative open-source 
firmware for ESP8266-based devices.
As of April 2020, there are currently over 1180 devices supported,\cite{tasdirec}
which also includes many commercial consumer electronics based on ESP8266 chip,
that can be disassembled and ``hacked'' by flashing the alternative firmware
(such devices, unfortunately, will lose their warranty).
The firmware provides all necessary functions and non-proprietary
interfaces for communication over the TCP/IP, utilizing multiple protocols
(e.g., HTTP, MQTT).


The difficulty of flashing the firmware differs for each device. The programming pin
on the ESP8266 chip must be pulled down to the ground and connected to a computer using
any compatible FTDI device. There are many existing tools (e.g., 
esptool.py\footnote{\href{https://github.com/espressif/esptool}{https://github.com/espressif/esptool}})
that provides simple and easy to use command-line
interface for flashing a new firmware to the device. Detailed description and
steps, how to prepare a ESP8266 device are mentioned in 
\nameref{installguide} on page \pageref{installguide}.

Devices equipped with Tasmota firmware can communicate over HTTP API or MQTT.

\hypertarget{x-results}{\section{Results}}
All three components of the system are implemented and specific hardware
for the testing environment was selected.


The communication server was created and provides live OpenAPI documentation.
A simple but sufficient web application for configuring the room for use with
OpenHVR was created. The web application is client-side and is included with
the OpenHVR server code. The server is hosting static files,
including the client-side application.
For using with Unity game engine, a set of scripts denoted as `Unity plug-in'
were also implemented. The devices communicate over HTTP API;
latencies of MQTT was unsatisfactory for the real-time connections, and Art-Net
did not meet the requirements for being a low-cost solution.

\begin{figure}[h]{}
\centering\includegraphics[width=\textwidth]{assets/running-server.png}
\caption{Screenshot of running server and the room configurator on the same machine}
\end{figure}

\begin{figure}[h]{}
\centering\includegraphics[width=\textwidth]{assets/configurator-ui.png}
\caption{Detailed screenshot of the resulting UI of room configurator}
\end{figure}

\begin{figure}[h]{}
\centering\includegraphics[width=\textwidth]{assets/openhvr-unity-plugin-usage.png}
\caption{Screenshot of Unity plug-in in use; the directional wind effect is set to simulate wind blowing from the virtual window.}
\end{figure}

\chapter{Example application using OpenHVR}

To demonstrate the abilities of the system, an example VR application
was created.


In the application, the user can walk around the virtual room and interact with
various items inside the room. Some of the items equip the OpenHVR
effects. Each example effect use is described below.


One of the user tests will use this example application, where
participants will be asked about an overall opinion on the experience and
how immersive it is.


The 3D models and materials were created using Microsoft Maquette
\footnote{VR prototyping tool, allowing a fast building of spatial environments \href{https://maquette.ms}{https://maquette.ms}}
and Unity’s XR Interaction Toolkit library provides the ability to
interact with various items inside the 3D worlds. This library allows users to
pick items with their hands and iteract with them intuitively (e.g., throwing them
or putting them somewhere).


The created models are very basic; there was not much of a focus on the design
of the models or making the scene photorealistic. Because of the limited time,
the scene design is functional and simplistic, not beautiful.


By making this example application, the OpenHVR project was validated and proved
to work as a whole and can be used for the desired purpose. The application uses
all parts of the system — the server must be running, the devices must be
connected and tracked their positions. The application is producing effects
using those devices, so the Unity plug-in must be used to produce the effects.


\begin{figure}[h]{}
\centering\includegraphics[width=2.5truein]{assets/test-room.png}
\caption{}

\end{figure}

\hypertarget{x-testing-entities}{\section{Testing entities}}
The example application could be used for user testing. For those purposes,
the application scene includes two main testing entities.


\hypertarget{x-windows}{\subsection{Windows}}
There are multiple breakable windows in the room. Each of the interactable
windows can be set to produce various combinations of effects. Any combination
of the following effects can be turned on or off, according to testing
scenario requirements:


\begin{itemize}

\item Visual effect of blowing wind

\item Sound effect of blowing wind

\item Blue cold screen colorization to indicate changing temperature

\item OpenHVR wind effect at the location of the window

\end{itemize}


The room also contains interactable pieces of stones. Users can grab those
stones and throw them at the windows to break them. After breaking the window,
the effects will run.


\begin{figure}[h]{}
\centering\includegraphics[width=2.5truein]{assets/test-room-windows.png}
\caption{}

\end{figure}

\hypertarget{x-fireplaces}{\subsection{Fireplaces}}
The virtual room will have multiple fireplaces. Users can grab a match and
start a fire. When a fire is started, similarily to the windows, an optional
combination of effects will start:


\begin{itemize}

\item Visual effect of fire

\item Sound effect of fire and cracking wood

\item Orange warm screen colorization to indicate the change of temperature

\item OpenHVR heat effect at the location of the fireplace

\end{itemize}


\begin{figure}[h]{}
\centering\includegraphics[width=2.5truein]{assets/test-room-fireplaces.png}
\caption{}

\end{figure}
\chapter{User testing}

User testing is divided into two parts. In the first part,
some testers are asked to help test and measure some properties
of the effect devices. In the second part, the OpenHVR example application is
presented to them, and they are asked on their opinion.


Even though the current pandemic situation made in-person testing a bit harder,
eight testers were willing to perform the tests, for which they are owed
a particular debt of gratitude.
The test was performed at the Faculty of Information Technology, using
the Oculus Rift VR headset with Oculus Touch tracked controllers.


\begin{figure}[h]{}
\centering\includegraphics[width=2.5truein]{assets/IMG_5238.jpeg}
\caption{}

\end{figure}

\hypertarget{x-devices-property-measuring}{\subsection*{Devices property measuring}}
Some testers were asked to help with measuring two properties of the fans used
for testing. The measurements will tell us how precisely the user can determine the
direction from where the wind is coming from and how long it takes until the
fan spins up and the user can feel the wind on his skin.


For the next two tests, the following conditions must be satisfied:


\begin{itemize}

\item Tester was introduced to the VR and the headset and controller
he or she will be using.

\item There are no objects or people in the play space that might negatively affect
the test process.

\item The room must have all doors and windows closed, without running A/C
or other potential sources of winds.

\end{itemize}


The test requires testers to know how to interact with objects and how to move
inside the virtual world. The virtual room is scaled appropriately to the
real room where the VR experience takes place, so the tester can walk around
and avoid using alternative methods of moving within the virtual world
(such as teleport, which can be confusing for first-time VR users).


Each device is different and depends on the power and quality of the fan.
As an example, a specific measuring tests were performed for a single fan
used for testing purposes:


\begin{center}
\begin{tabular}{|c|c|}
\hline
Manufacturer & Sencor \\ 
Model & SFN 4047WH \\ 
Rating & AC 220-240 V, 50 Hz \\ 
Consumption & 50 W \\ 
Loudness & 50 dB \\ 
Fan radius & 40 cm \\ 
Maximum airflow volume & $ 49 \frac{m^3}{min} $ \\ 
Maximum airflow speed & $ 2,9 \frac{m}{s} $ \\ 
\hline
\end{tabular}
\end{center}

The results of these measurements will serve as reference values for future
room set-ups. It is evident that for different devices, these measurements
would need to be retaken.


\hypertarget{x-fan-spin-up-time}{\subsubsection*{Fan spin-up time}}
It takes a noticeable amount of time until the fans start spinning and
blow wind. It is necessary to delay the in-game effects by an offset that
will make the in-game visual or sound effect in sync with the effect
produced by the fan.
For this test, the mentioned fan was used, positioned straight in front of
testers, with 1.5 meters distance and 1.5 meters height level.


Testers were asked to follow the instructions on-screen: first press a button
when they are ready and press it again the exact moment they can feel a wind
blowing on their skin. The test program records the time between the presses.
Each tester collected four samples.


Throughout the testing, the VR headset blocks the tester’s view (an arbitrary
background image is shown in the view), and loud music plays, so the tester
cannot see and hear the fan spinning. After the tester first presses the button,
a random fan will start to spin, and stopwatch is started. After the second
press, stopwatch and fan will stop, and the tester will wait for a few seconds
to repeat the process for all positions.


There was a total of 16 samples collected.


\begin{center}
\begin{tabular}{|c|c|c|c|}
\hline
Tester B & 2709 ms & 3704 ms & 3035 ms \\ 
Tester C & 2722 ms & 3953 ms & 3110 ms \\ 
Tester D & 2432 ms & 2908 ms & 2682 ms \\ 
Tester H & 2566 ms & 3676 ms & 3068 ms \\ 
\hline
\end{tabular}
\end{center}

\begin{center}
\begin{tabular}{|c|c|}
\hline
Minimum & 2432 ms \\ 
Maximum & 3953 ms \\ 
Mean & 2974.188 ms \\ 
Std. deviation & 420.202 ms \\ 
Median & 2865 ms \\ 
\hline
\end{tabular}
\end{center}

\begin{figure}[h]{}
\centering\includegraphics[width=2.5truein]{assets/IMG_5242.jpeg}
\caption{}

\end{figure}

\hypertarget{x-fan-spread-range}{\subsubsection*{Fan spread range}}
All fans are directional. The objective of this test is to assess
the optimal number of fans that would be necessary to build a full 360-degrees
set-up of fans surrounding the players. By analyzing the measurements,
we will try to determine, at what point the testers could not distinguish
between positions of the fans.


The fans are positioned at a distance of 1.5 meters from the tester.
Considering the tester is looking forward, the fans will
have four different positions:


\begin{itemize}

\item \textbf{A:} Placed directly in front of the tester

\item \textbf{B:} Rotated 15 degrees clockwise from the tester’s forward look

\item \textbf{C:} Rotated 45 degrees clockwise from the tester’s forward look

\item \textbf{D:} Rotated 90 degrees clockwise from the tester’s forward look

\end{itemize}


The positions are indicated in the following diagram:


\begin{figure}[h]{}
\centering\includegraphics[width=2.5truein]{assets/fans-positions.pdf}
\caption{}

\end{figure}

The distribution of the positions is determined by the limited number of
available effect devices. It tests the most interesting cases — 90 degrees between A and D is a large distance, and it is expected that testers
will be able to distinguish the difference without any problems.
On the other hand, the 15 degrees between
A and B at the 1.5m distance will probably not be distinguishable.



The scenario is following: Testers are asked to wait until a fan start spinning
and then determine from where is the wind blowing. By pointing there with the
tracked VR controllers and pressing the button, they save their estimate.
While the tester is asked to determine the fan’s position, the VR headset
blocks his view and the test program will play loud music, so they
cannot locate the fan with hearing.


After pressing the button, the test program will save the following information:


\begin{itemize}

\item Three-dimensional direction vector oriented from controller position to
the current spinning fan location (reference input)

\item Three-dimensional direction vector representing the rotation of the
controller (tester’s input)

\item The magnitude of the difference between those two mentioned vectors

\end{itemize}


Testers were asked to measure the A position three times (without their knowledge),
so each tester collected following six samples for fan positions: A, B, A, C,
A, and D. Thanks to this, we can verify the repeatability of successful
localization of the same fan at the same position.


For each of the sample collection, the test program collects the following vectors:


\begin{enumerate}

\item{Normalized location difference \texttt{l} between the current world position of the right
controller and location of the currently blowing fan. The difference gives a direction
vector from the controller to the fan.}

\item{Direction vector \texttt{r} of current controller rotation.}

\end{enumerate}


For each pair of these two vectors, the magnitude of their difference is calculated
as \texttt{d = |(r-l)|}.
Lower values represent greater tester’s precision (value \texttt{0} represents
perfect alignment).



Unfortunately, the results collected at the user testing sessions
\textbf{do not seem to bear the desired informational values}.
There was an error in the execution of the test. It is suspected that
the tracking and real world’s mapping to the virtual world were done
incorrectly, and therefore the results do not give the correct answers to
questions that led to this test. \hyperlink{15-attachments#att2}{}
contains the collected data.


The correct determination of wind source height was not the primary
objective to test. Because testers were not correctly instructed on this matter,
after the test, some of them reported that they intentionally ignored the
height, and some of them tried to determine the height correctly.
To fix the measured data, the upward axis information from the vectors was
removed, and data are analyzed only on 2D-plane (view from top).
Conveniently this makes the data easier to visualize, but even this
processing was not able to fix the error in measured data.


It is not possible to draw any conclusion from this test. Theoretically, the
test should be valid and correct execution of the test should give the
desired answers. Unfortunately, the execution while user testing was not
correct, and this test failed and must be performed again.


\hypertarget{x-main-user-test}{\subsection*{Main User Test}}
For the primary test, an example app was used, with four breakable windows
and three fireplaces (lower number of the fireplaces was caused by
the headset tracking limitations). All 8 of the testers were participating
in this test.


\hyperlink{09-example-app}{} describe the appearance of the room and the entities in it.


\hypertarget{x-scenario}{\subsubsection*{Scenario}}
Using the example app (as described in the previous chapter), testers will
be asked to use the application to interact with the virtual windows and
virtual fireplaces.


There are conditions, that needs to be met before the test starts:


\begin{itemize}

\item Tester was introduced to the VR and controls of
the example app (e.g., how to pick objects, what buttons to use)

\item There are no objects or people in the play space that might negatively affect
the test process

\item The room must have all doors and windows closed, without running A/C
or other potential sources of winds

\end{itemize}


There will be four variants of interactable windows, and each of them will
behave differently; users will be asked to compare the differences between
them and overall evaluate relatively the quality of the effects and
immersion into the virtual world.


\begin{itemize}

\item Window A — only visual effect of blowing wind

\item Window B — visual and sound effect of blowing wind

\item Window C — visual and sound effect of blowing wind + temperature change
indication by coloring the screen with blue color

\item Window D — visual and sound effect of blowing wind + OpenHVR external effect
using pedestal fan, simulating actual blowing wind from the position of
the window

\end{itemize}


After breaking all of the four windows, users will be instructed to evaluate
the effects and then asked to continue similarily with the fireplaces.


Three variants of the fireplaces will be similar to windows:


\begin{itemize}

\item Fireplace E — visual and sound effect of burning fire

\item Fireplace F — visual and sound effect of burning fire + temperature change
indication by coloring the screen with orange color

\item Fireplace G — visual and sound effect of burning fire + OpenHVR
external effect using the infrared heater, simulating actual heat coming
out of burning fire’s position

\end{itemize}


After performing all the tasks, users will be asked to evaluate the
effects of burning fire and additional questions.


\hypertarget{x-questions}{\subsubsection*{Questions}}
\begin{itemize}

\item "On a scale of 0-10 (0 worst, 10 best) evaluate the level of immersion and quality
of effects in the virtual environment when breaking the window, relatively
between the windows A/B/C/D."

\item "On a scale of 0-10 (0 worst, 10 best) evaluate the level of immersion and quality
of effects in the virtual environment when lighting up the fireplace, relatively
between the fireplaces E/F/G."

\end{itemize}


\hypertarget{x-additional-questions}{\subsubsection*{Additional questions}}
\begin{itemize}

\item "Did the wind effect at window D affect your comfort while using VR?"

\item "Did the heat effect at fireplace G affect your comfort while using VR?"

\item "Did the special effects (at D or G) negatively affected your experience
compared to effects without using the OpenHVR system (A-C, E, F)?"

\item "Do you have any suggestions? What were you thinking about while performing
the test?"

\end{itemize}


\hypertarget{x-results}{\subsubsection*{Results}}
All testers successfully performed the test. Some of them asked to
restart the test because they were not able to focus on correct objects,
forgotten the differences, or simply were not able to evaluate the
differences properly. Re-running of the test does not seem to affect the
test results negatively.


\begin{figure}[h]{}
\centering\includegraphics[width=2.5truein]{assets/IMG_5249.jpeg}
\caption{}

\end{figure}

The following table lists collected answers on questions about immersion level and
quality of the effects.


\begin{center}
\begin{tabular}{|c|c|c|c|c|c|c|c|}
\hline
Tester A & 2 & 5 & 10 & 7 & 3 & 6 & 10 \\ 
Tester B & 0 & 0 & 4 & 8.5 & 0 & 4.5 & 9.5 \\ 
Tester C & 0 & 2 & 5 & 8 & 2 & 5 & 9.5 \\ 
Tester D & 0 & 5 & 5 & 10 & 5 & 5 & 10 \\ 
Tester E & 1 & 2 & 2 & 7 & 1 & 1 & 6 \\ 
Tester F & 1 & 3 & 3 & 6 & 3 & 3 & 6 \\ 
Tester G & 2 & 4 & 5 & 8 & 4 & 4 & 9 \\ 
Tester H & 0 & 3 & 4 & 9 & 2 & 4 & 10 \\ 
\hline
\end{tabular}
\end{center}

Apart from the numeric results, users were asked additional questions.
None of the users reported any negative effects while experiencing the special
effects.


\begin{figure}[h]{}
\centering\includegraphics[width=2.5truein]{assets/vlcsnap-2020-05-22-16h41m28s649.jpg}
\caption{}

\end{figure}

Following is the list of relevant comments, that some testers reported
(loosely translated from the Czech language):


\begin{itemize}

\item "I have noticed that the smoke was coming out of the walls, which
lowered the immersion for me."

\item "At window C, I thought I had felt a little wind,
that was not at all corresponding to what I saw."

\item "Because of the visual effect, I was expecting a much stronger wind
effect."

\item "I think the fan in the real world was at the wrong location from the
window in the virtual world."

\item "I saw the infrared heater before the beginning of the test, and I was
a little scared that I might get burned."

\item "I did not see any difference between B and C. I have just noticed that
some of them had glass shards and some not."

\item "I did not like the quality level of graphics."

\item "The infrared heater positively surprised me."

\item "I felt like the sound was coming from different places than they actually
were at."

\item "The windows were much more interesting and entertaining, but the fireplace
was much more believable and immersive."

\end{itemize}


Overall, users were satisfied with the experience. Some of the users reported
that they did not like the graphical appearance of the environment.
Unfortunately, most of the users were not able to distinguish the difference
between windows B and C, and fireplaces A and B. Almost everyone reported
difficulties with this matter.

\setsecnumdepth{part}
\chapter{Conclusion}

{TODO} One of the unique properties of the work is the amount of technologies
and programming languages used at once. The project covered working with
C\#, JavaScript, Go …​


{TODO}


\hypertarget{x-complications}{\section{Complications}}
After a practical experience with the Beego framework, it cannot be said that
it was the best choice. The framework’s design isn’t ideal and it seems, that
overall the quality is low (for example there are severe spelling mistakes in
error messages). Compared to other frameworks, the ORM funcionalities are very
limited and was barely suitable for such simple application. Lots of
functionalities was not working, even though documentation claims it should
work. For example, the relationships between tables are broken and sometimes
crashes the server, so they couldn’t be used. It took a big chunk of time
to troubleshoot and debug such situations. Sadly, the framework didn’t make
the development that much easy and comfortable as anticipated.


\chapter{Future works}\label{futureworks}

The main appeal for follow-up work is to create an actual VR application
utilizing the OpenHVR project practically. The example application presented 
in this work is only for testing and presentation purposes and is not considered
as a full VR experience with enriched environmental effects. To present truly
immersive experience, the users expect excellent graphical quality of the 
environment.

It also might be interesting to think about configurable limitations specified
by the administrator.
Some effects might consume a considerable amount of electricity (especially
true for space heaters), and currently, developers designing the effects are
not limited in how many effects they can turn on at the same time and for how
long.

The failed execution of the fan direction test should be redone. The test
should give the desired results by paying attention to accurate tracking and
precise mapping real-world devices to virtual coordinates.

\section{WebXR}

WebXR capabilities could be incorporated into the Configurator Tool.
Users would not need to use the Unity Helper for Configurator to mark positions
of effect devices accurately. He/she would see the room
size and properties in the Web Configurator in real-time. Using a headset or
just the controllers, he/she would define the effect devices' positions and
rotations the same way as with the plug-in. 

On the other hand, the support in browsers for WebXR is still very experimental,
and more importantly, it is not guaranteed that the space coordinate frame
of WebXR applications will be the same as the space coordinate frame of OpenVR
or other runtime applications.

\pagebreak

\section{Service discovery}

Unity Plug-in should be able to discover the OpenHVR server automatically. 
This feature would mitigate the necessity to manually specify the IP address
of the computer running the OpenHVR server instance. Such a feature might
significantly enhance the user experience.
\chapter{Install Guide}

The purpose of this chapter is to guide administrators through the process
of instalation of the OpenHVR system.

\section{Server installation}

Users are offered two ways to install OpenHVR server. Installation of the
server contains the configuration tool and is not needed to install separately.

\subsection{Install using Docker image}

The easiest way to install OpenHVR server is to run Docker image, which is
publicly available in the Docker Hub\footnote{https://hub.docker.com/}
container registry. The requirement is to have Docker Engine of version 2.2 or
newer installed on your computer. 

Run the image using the following command:

\begin{verbatim}
    docker run -d --mount \
    source=od,destination=/go/src/github.com/mmajko/openhvr-server/_data \
    -p 47023:47023 marianhlavac/openhvr
\end{verbatim}

The command will run the server on port 47023, and the configuration tool will
be available at \verb|http://localhost:47023|

\subsection{Install manually}

If you do not prefer Docker, you can compile the server manually and run it.

You will need to install the following requirements:

\begin{itemize}
    \item Go Tool command-line utility (version 1.13 or newer)
    \item Node.js JavaScript runtime (version 13.11 or newer) 
    \item NPM package manager (version 6.14 or newer)
\end{itemize}

To learn how to obtain an installation of Go tools, refer to the
\hyperlink{https://golang.org/doc/install}{Installation Instructions} at the
official Go website.\footnote{https://golang.org is the official Go website.}

Navigate to the directory containing the OpenHVR server source code named
\verb|openhvr-server|. In the directory run the following command to build the
server executable:

\begin{verbatim}
    go build main.go
\end{verbatim}

The library dependencies required for the build will be automatically downloaded.

After the build process is finished, you can find an executable in the current
directory.

Go to the \verb|configurator-tool| directory, which can be found in the current
working directory, and install the configurator tool dependencies using the npm's
\verb|install| command,

\begin{verbatim}
    npm install
\end{verbatim}

and build it with following command:

\begin{verbatim}
    npm run build
\end{verbatim}

The built files will appear in the \verb|./public/build| directory.

Before running the server, make sure the \verb|_data| directory exists inside
the \verb|openhvr-server| directory, and run the compiled executable.

\begin{verbatim}
    ./main
\end{verbatim}

\section{Effect device preparation}

Before your electrical appliance can be used for producing the effects
within the OpenHVR system, it needs to be configured first.

\textbf{NOTE:} Make sure that the device is connected properly. Only a qualified
person should be allowed to install electronics working with higher voltages, to avoid
damaging the equipment or other risks.

\subsection{Flash the Tasmota firmware}

The device must be flashed with the Tasmota firmware on the chip of the
device.
The appliance itself must be based on the ESP8266 chip, or connected to a 
smart relay or switch, which is based on the ESP8266 chip. To check if your
device is compatible with Tasmota firmware, consult the 
\hyperlink{https://templates.blakadder.com}{Tasmota Devices Templates Repository}
which lists all supported devices.

\textbf{TIP:} Some devices are difficult to flash, requires soldering on the
pins of the chip or other sophisticated methods of flashing. If you are not
experienced with electronics, leave the flashing process to someone
more experienced or try searching for a pre-flashed devices on the market.

\subsection{Add the device to OpenHVR}

After your device is ready and running the Tasmota firmware, it needs to be
connected to the same LAN as the OpenHVR server. Determine its IP address and
open OpenHVR configuration tool.

Press the \textit{Add Device} button, select the type of effects this device
will be producing, and enter the connection details of your device into
the form. If you do not yet know the position and rotation of the device, you
can keep the fields empty. Select appropriate driver for controlling the
appliance. Specify connection parameter, if neccessary. If you have multi-channel
Tasmota device, the parameter must be equal to the channel name 
(e.g., \verb|Power1| for the first channel, \verb|Power2| for the second one).
\chapter{Developer Guide}

This chapter briefly describe how developers can extend the functionalities
of the OpenHVR system, or use the Unity plug-in to fit the effects 
into their virtual reality applications and games.

To understand some of the server code, an experience with the Beego framework
is required. To learn more about the Beego framework, visit the
\hyperlink{https://beego.me/docs/intro/}{Beego Documentation} on their official
website.\footnote{https://beego.me is the official Beego framework website.}

\section{Writing new device driver}

Directory \verb|openhvr-server/devicedrivers| contains implementations of
device drivers. To add new drives, create a new file with a new function,
that conform to following interface:

\begin{verbatim}
    package devicedrivers

    import "github.com/mmajko/openhvr-server/models"

    func FooDriver(device *models.Device, request *models.EffectRequest) error {
        ...
    }
\end{verbatim}

Implemented driver then must be registered in file \verb|openhvr-server/main.go|
on line 26. Example:

\begin{verbatim}
    ...
    devicedrivers.RegisterDriver("name_of_the_driver", devicedrivers.FooDriver)
    ...
\end{verbatim}

\section{Server API documentation}

The Beego framework can automatically generate API developer documentation.
To read the documentation, run the developer server using the \verb|bee| command
(can be installed using command \verb|go get -u github.com/beego/bee|)

Run the developement server with the \verb|-downdoc=true -gendoc=true| arguments

\begin{verbatim}
    bee run -downdoc=true -gendoc=true
\end{verbatim}

then visit the address \verb|http://localhost:47023/docs| to read the
documentation.

The generated documentation can be also found on the enclosed CD of the thesis.

\section{Using the Unity plug-in}

Download the Unity package file \verb|openhvr.unitypackage| from the GitHub
repository or enclosed CD.

In Unity Editor, open contextual menu and select 
\verb|Assets| > \verb|Import Package| > \verb|Custom package...| and select
the \verb|openhvr.unitypackage| file.

In the scene, where the OpenHVR effects are desired to be used, create an empty
game object and add a new \verb|OpenHVR Manager| component to it.

Test the functionality by turning on the debug mode in the manager and running
the game. In the console a message will appear, reporting if the connection
to the OpenHVR server is successful.

To display OpenHVR devices and access their locations from in-game, attach
the \verb|OpenHVR Devices Enumerator| component to any game object.

To run a OpenHVR effect, attach the \verb|OpenHVR Effect Source| component to
any game object. For each effect you can set the following properties:

\begin{description}
    \item[Duration] The duration of the effect in seconds.
    \item[Effect Type] Type of the effect to produce.
    \item[Range] Determines the size of the sphere representing the effect range.
    If you want to turn on multiple devices at once, you can make the range larger.
    \item[Directional] If turned on, the effect will be directional and will
    be applied only to devices in the correct direction.
    \item[Play on Awake] The effect will start when the game object is created.
\end{description}

The position and direction of the effect source is determined by the values
present in the \verb|Transform| component of your game object.

To control the effect programatically, a following public methods in the
\verb|OpenHVREffect| class can be accessed:

\begin{description}
    \item[void Play()] Starts the effect for the specified duration.
    \item[void Cancel()] Immediately cancel the effect.
    \item[bool IsPlaying()] Returns \verb|true| if the effect is currently playing.
\end{description}
\chapter{Attachments}

\hypertarget{x-attachment-1:-measurements-of-the-"fan-spin-up"-test}{\section*{Attachment 1: Measurements of the "fan spin-up" test}}

\hypertarget{x-attachment-2:-results-of-the-failed-"fan-direction"-test}{\section*{Attachment 2: Results of the failed "fan direction" test}}
Following table contains the calculated magnitudes of vector differences \texttt{d}
for each collected user’s sample, originally collected with the
height information (Y-axis not removed)


\begin{center}
\begin{tabular}{|c|c|c|c|c|c|c|}
\hline
Tester A & 0,1607147 & 0,6379536 & 0,305311 & 0,3321228 & 0,8487036 & 1,263571 \\ 
Tester B & 0,3658605 & 0,4990586 & 0,1040664 & 0,9002404 & 0,5659719 & 1,285348 \\ 
Tester C & 0,298934 & 0,3214657 & 0,3184844 & 0,1148799 & 0,159058 & 1,06579 \\ 
Tester D & 0,7013891 & 0,7075886 & 0,7987939 & 0,4982374 & 0,6480926 & 0,7447294 \\ 
Tester H & 0,3276841 & 0,352311 & 0,3917351 & 0,1454643 & 0,4060867 & 0,2452434 \\ 
\hline
\end{tabular}
\end{center}

Colors define the target fans written in uppercase letters and colors
(red fan A, green fan B, blue fan C, purple fan D),
lowercase letters stands for individual testers and their results. This data
have Y-axis information removed and is visualized on a 2D-plane
(viewed from top).


\begin{figure}[h]{}
\centering\includegraphics[width=\textwidth]{assets/direction-results.pdf}
\caption{}

\end{figure}



\bibliographystyle{template/iso690}
\bibliography{bibliography}

\setsecnumdepth{all}
\appendix

\chapter{Acronyms}
% \printglossaries
\begin{description}
	\item[GUI] Graphical user interface
	\item[XML] Extensible markup language
\end{description}


\chapter{Contents of enclosed CD}

%change appropriately

\begin{figure}
	\dirtree{%
		.1 readme.txt\DTcomment{the file with CD contents description}.
		.1 exe\DTcomment{the directory with executables}.
		.1 src\DTcomment{the directory of source codes}.
		.2 wbdcm\DTcomment{implementation sources}.
		.2 thesis\DTcomment{the directory of \LaTeX{} source codes of the thesis}.
		.1 text\DTcomment{the thesis text directory}.
		.2 thesis.pdf\DTcomment{the thesis text in PDF format}.
		.2 thesis.ps\DTcomment{the thesis text in PS format}.
	}
\end{figure}

\end{document}
